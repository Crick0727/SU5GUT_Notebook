% ---------------------------------------
%
%  Some difficulties of Standard Model
%  SM_problems.tex
%  Program modified by Yasutoki Takamura
%  Last Modified Jul 20 2024
%
% ---------------------------------------
\section{標準模型の抱える問題}
素粒子標準模型は高エネルギー物理学の実験をほぼ正確に予言することができるため, 大きな成功を収めた.
特に2011年にCERNにある大型ハドロン衝突型加速器(Large Hadron Corrider; LHC)が標準模型に現れるHiggs粒子を発見したことにより, 標準模型は揺るぎないものとなった.
しかし, 次のような課題があり, 理論の拡張が迫られている.
\begin{itemize}
        \item ニュートリノ質量, およびニュートリノ振動\\
              標準模型ではニュートリノは質量を持たない粒子として存在する.
              しかし1998年にニュートリノ振動がスーパーカミオカンデで観測されたことにより, ニュートリノは質量を持つことが示唆されたため, 標準模型を何かしら拡張する必要があると考えられている.
      \item 重力相互作用
      \item 真の統一理論
      \item 電荷の量子化\\
            素粒子の電荷は単位電荷の整数倍の値を持つ.
            非可換群の固有値であれば量子化が実現できるが, ハイパーチャージ$Y$は可換群である$U(1)$対称性における無限小演算子であり, 固有値$Y$量子化は行えない.
            したがって標準模型で電荷$Q$は$Q = I^3 + \frac{Y}{2}$という関係に基づいて決定されるが, この電荷が量子化される根拠は標準模型に存在しない.
      \item 階層性問題
      \item 予言できないパラメーターの数
      \item 暗黒物質
      \item 暗黒エネルギー
\end{itemize}
%EOF

