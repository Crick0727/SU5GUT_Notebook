\section{大統一理論}
大統一理論はH.GeorgiとS.L.Glashowにより1974年に提唱された\cite{PhysRevLett.32.438}.
大統一理論では重力を除いた3つの相互作用を1つに統一することを目的としている.
したがって, ゲージ対称性は単純群によって記述されると考えられており, 標準模型のゲージ群である$\mathcal{G}_\text{SM}= SU(3)_c\times SU(2)_L\times U(1)_Y$を部分群として内包する群を考える.
このことから, 前節で述べられているような標準模型の問題点はいくつか解決されると考えられている.

\textcolor{red}{要修正: どの点で何が解決できているのか}

はじめに, $SU(5)$大統一理論が最小模型である理由は, 標準模型のゲージ対称性である$SU(3)_c\times SU(2)_L\times U(1)_Y$が, $\text{rank}=4$であり, これを内包できる最小の単純群が$SU(5)$であるためである.
特に, $SU(3)$のもつ3重項をもち, $SU(2)$がもつ2重項の複素表現をもつ性質を考えると$SU(5)$群を用いて, 最小模型を構築する.


\subsection{Georgi-Glashow モデル}
ここから$SU(5)$群を考える.
ゲージ群を決定すると, ラグランジアンに導入する場の既約表現を指定すれば理論を定めることができる.
$SU(5)$の基本表現は$5$表現であるから, 

この模型では, 左巻きフェルミオンは世代ごとに次の二つの表現に当てはめて考える.
\begin{align}
 \bar{\bm{5}}=\begin{pmatrix}
    d_1 ^c \\
    d_2 ^c \\
    d_3 ^c \\
    l      \\
    -\nu_l
    \end{pmatrix},\qquad
\bar{\bm{10}}=\begin{pmatrix}
         0 &  u_3^c & -u_2^c & u_1 & d_1 \\
    -u_3^c &      0 &  u_1^c & u_2 & d_2 \\
     u_2^c & -u_1^c &      0 & u_3 & d_3 \\
    -u_1   &   -u_2 &   -u_3 &   0 & e^c \\
      -d_1 &   -d_2 &   -d_3 &-e^c &   0 \\
    \end{pmatrix}\label{GUT-1}
\end{align}
ただし, $x^c$は$x$を同じカイラリティで荷電共役したものを表している.
この$x$の例として, 左巻きの電子$e^-_L$を考える.
荷電共役変換である$(e^-_L)^c$を考えると, これは右巻きの電子$e_R^-$となるが, 左巻きの陽電子である$e_L^+$と同じである.


