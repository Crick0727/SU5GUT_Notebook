% ----------------------------------------
%
%  Grand Unified Theory
%  grand_unified_theory.tex
%  Program modified by Yasutoki Takamura
%  Last Modified Jan 27 2025
%
% ----------------------------------------
\chapter{SU(5)大統一理論}
大統一理論はH.GeorgiとS.L.Glashowにより1974年に提唱された\cite{PhysRevLett.32.438}.
大統一理論では重力を除いた3つの相互作用を1つに統一することを目的としている.
したがって, ゲージ対称性は単純群によって記述されると考えられており, 標準模型のゲージ群である$\mathcal{G}_\text{SM}= SU(3)_c\times SU(2)_L\times U(1)_Y$を部分群として内包する群を考える.
このことから, 前節で述べられている標準模型の問題点のうち,
\begin{itemize}
  \item 電荷の量子化
  \item 予言できないパラメータの数
\end{itemize}
は大統一理論によって解決される可能性がある.

大統一理論にも様々な単純群を考えることが可能であり, $SU(5)$や$SO(10)$, $E_6$などの模型を考えることが可能であるが, このノートでは最小模型である$SU(5)$について取り扱い, 理論の拡張を試みる.

はじめに, $SU(5)$大統一理論が最小模型である理由を考える.
標準模型のゲージ対称性である$SU(3)_c\times SU(2)_L\times U(1)_Y$は, 群の階数が$\text{rank}=4$であり, これを内包できる群を考える必要がある.
その中で考えられる単純群は, $\text{O}(8)$, $\mathrm{O}(9)$, $\mathrm{Sp}(8)$, $\mathrm{F}_4$, そして $\mathrm{SU}(5)$である.
このうち, $SU(3)$のもつ3重項をもち, $SU(2)$がもつ2重項の複素表現をもつという性質は$SU(5)$群を用いて記述することができるため, 最小模型として$SU(5)$群を考えて理論を構築することが可能となる.

%\section{Georgi-Glashow モデル}
\section{フェルミオンの表現}
ここでは$SU(5)$群において標準模型に現れるフェルミオンがどのように当てはめられるのか考える.

ゲージ群を決定すると, ラグランジアンに導入する場の既約表現を指定すれば理論を定めることができる.
$SU(5)$の基本表現は$5$表現であり, $5$表現か, その複素表現である$\overline{5}$を用いることですべての表現を構成することができる.
$SU(5)$群の階数は4であるから, 生成演算子を$L_i$$(i=1,\cdots,24)$ のうち対角化可能なものが4つ存在する.
$SU(3)_c$の対角化可能な生成演算子を$\lambda_3,\lambda_8$, $\mathrm{SU}(2)$のものを$I_3$ (アイソスピン), $U(1)$の超電荷を$Y$に対応させて考えることができる.

次にフェルミオンの次元を$\mathrm{SU}(3), \mathrm{SU}(2)$の表現次元, $U(1)$の超電荷$Y$で $(1, 2, -1)$のように表し, どのように$SU(5)$模型に当てはめられるか考える.
\begin{center}
\begin{tabular}{clc}
             &                         & 多重度 \\
  $(u, d)_L$ & $(3  , 2, \frac{1}{6})$ &      6 \\
    $d_L^c $ & $(3^*, 1, \frac{1}{3})$ &      3 \\ 
    $u_L^c $ & $(3^*, 1,-\frac{2}{3})$ &      3 \\ 
$(\nu, e)_L$ & $(1  , 2,-\frac{1}{2})$ &      2 \\ 
     $e_L^c$ & $(1  , 1,           1)$ &      1 \\ 
   $\nu_L^c$ & $(1  , 1,           0)$ &      1 \\ 
\end{tabular}
\end{center}

基本表現をこの中から選ぶことで
\begin{align}
  (q_L^c, \nu_L, e_L) = (3^*, 1, Y)\oplus\left(1,2,-\frac{1}{2}\right) = \overline{5}
\end{align}
と考えられる.
ここではまだ$q$の正体が$u, d$クォークのどちらかは定かではないが, $\mathrm{SU}(5)$群のトレースは0でなければならないため, $Q$を電荷とすると上記の電荷は
\begin{align}
  \sum_{a=1}^5 Q_a = 3Q_{qc}+Q_\nu + Q_e\nonumber\\
  \therefore\quad Q_q = -Q_{qc} = -\frac{1}{3}\label{quantum_Q}
\end{align}
となる.
したがって, $q=d$と選択することにより, 次のように$\bm{5}$表現を決めることができる.
\begin{align}
 \psi_{\overline{\bm{5}}} =\begin{pmatrix}
    d_1 ^c \\
    d_2 ^c \\
    d_3 ^c \\
    l      \\
    -\nu_l
  \end{pmatrix}=\left({3}^*,1,\frac{1}{3}\right)\oplus \left(1,2,-\frac{1}{2}\right)\label{GUT-5rep}
\end{align}
これによりクォークの電荷が荷電レプトンの$\frac{1}{3}$の整数倍となっていることが説明できる.
残りのフェルミオンを次元の大きい表現に当てはめることを考える.
基本表現よりも大きな表現は, すべて基本表現の積で表すことができる.
${\bm{10}}$表現は
\begin{align}
  \bm{5}\otimes\bm{5} = \bm{15}\oplus\bm{10}\nonumber
\end{align}
として構成することができる.
この$\bf{10}$表現は$\bm{10}\rightarrow (3^*,1,-\frac{2}{3})\oplus(3,2,\frac{1}{6})\oplus(1,1,1)$
と分解できるので, 次のように当てはめることができる.
\begin{align}
\psi_{{\bm{10}}} = \frac{1}{\sqrt{2}}\begin{pmatrix}
         0 &  u_3^c & -u_2^c & u_1 & d_1 \\
    -u_3^c &      0 &  u_1^c & u_2 & d_2 \\
     u_2^c & -u_1^c &      0 & u_3 & d_3 \\
    -u_1   &   -u_2 &   -u_3 &   0 & e^c \\
      -d_1 &   -d_2 &   -d_3 &-e^c &   0 \\
    \end{pmatrix} = u^c\left(3^*,1,-\frac{2}{3}\right)\oplus q\left(3,2,\frac{1}{6}\right)\oplus e^c(1,1,1)\label{GUT-10rep}
% Note
% 10表現であれば内部に規格化定数を入れる場合があるがここで必要ではないのか確認しなければならない.
% Dec 10: up quark massを出す際に規格化されている必要があるので, この規格化定数は必要.
\end{align}
ただし, $x^c$は$x$を同じカイラリティで荷電共役したものを表している.
この$x$の例として, 左巻きの電子$e^-_L$を考える.
$e^-_L$に対して荷電共役変換を行った$(e^-_L)^c$を考える.
これは右巻きの電子$e_R^-$となるが, 左巻きの陽電子である$e_L^+$と同じである.
\section{ゲージ粒子の表現}
ゲージ粒子を$SU(5)$大統一理論で考えるために, 群の生成子について考える.
一般に$SU(n)$ゲージ群の変換は, 
\begin{align}
  U =\exp\left(-i \sum_{i=1}^{n^2-1}\beta_i L_i\right)=\exp(-\bm{\beta}\cdot \bm{L})
\end{align}
で表される.
ここで, $L_i$は群の生成子であり, エルミート性があり, トレースレス($\mathrm{tr} L_i =0$)である.
これにより$U$はユニタリーであり, $\mathrm{det}U =1$を満たす.
$L_i$は
\begin{align}
  \mathrm{tr}(L_iL_j) = \frac{1}{2}{\delta_{ij}}\label{reguralization}
\end{align}
と規格化する.
ここで, 表現行列を昇降演算子との類推で今後のために
\begin{align}
  (L^a_b)^c_d \equiv \delta^c_b\delta^a_d -\frac{1}{n}\delta^a_b\delta^c_d,\quad ((L^a_b)^\dagger=L^b_a)
\end{align}
と定義する.
交換関係は
\begin{align}
  [L^a_b,L^c_d] = \delta^a_bL^c_b - \delta^c_bL^a_d
\end{align}
を満たす.
これまで$n$表現について考えたが, 共役な表現$\overline{n}$を考えた場合,
\begin{align}
  L^a_b(\overline{n}) = -L^{aT}_b = -L^b_a
\end{align}
を満たす.

ここから具体的にゲージ場について考える.
$SU(n)$ゲージ理論では, $n^2-1$個のゲージ場が存在する.
これらは随伴表現で表される.
$SU(5)$群の生成演算子をこれまでのように$[L_i]^a_b\,(i=1,\cdots,24, a,b=1,\cdots,5)$として構成する.
基本表現である$5$表現の類推からは$a,b=1,2,3$はカラー量子数, $a,b=4,5$がアイソスピン量子数と考えられる.
そのため$i=1,\cdots,8$を$SU(3)$部分, $i=9,10,11$を$SU(2)$の部分群として考えると
\begin{align}
  L_i(i=1,\cdots,8) = \left(\begin{array}{@{}c|c@{}}
      \mbox{$\cfrac{1}{2}\lambda_i$} &
      \begin{matrix}
        0 & 0 \\
        0 & 0 \\
        0 & 0 
      \end{matrix}\\
      \hline
      \begin{matrix}
        0 & 0 & 0\\
        0 & 0 & 0
      \end{matrix} &{\large{\mbox{$0$}}}
  \end{array}\right),\quad
  L_i(i=9,10,11) = \left(\begin{array}{@{}c|c@{}}
      \mbox{\Large 0} &
      \begin{matrix}
        0 & 0 \\
        0 & 0 \\
        0 & 0 
      \end{matrix}\\
      \hline
      \begin{matrix}
        0 & 0 & 0\\
        0 & 0 & 0
      \end{matrix} & {\mbox{$\cfrac{1}{2}\sigma_j$}}
  \end{array}\right)
\end{align}
ただし, $\sigma_{j=i-8}\,(j=1,2,3)$である.

対角行列は$L_3, L_8, L_{11}$にそれぞれ$SU(3), SU(2)$群の対角行列を対応させる.
残りの1つは$L_{12}$であり, $U(1)_Y$の超電荷を対応させる.
基本表現である$\bm{5}$表現の超電荷を対応させ, 規格化条件である式(\ref{reguralization})を用いることで
\begin{align}
  L_{12}=\frac{1}{2\sqrt{15}}\mathrm{diag}(-2,-2,-2,3,3)\nonumber
\end{align}
と決めることができる.


24個のゲージボゾンは随伴表現として存在し, 次のように分解できる.
\begin{align}
  \bm{24} = G(8,1,0)\oplus W(1,3,0)\oplus B(1,1,0)\oplus \overline{X}(3,\bar{2},-\frac{5}{6})\oplus X(\bar{3},2,+\frac{5}{6})
\end{align}
ゲージボゾンを$V_\mu$とすると,
\begin{align}
  \frac{1}{\sqrt{2}}V_\mu &= \sum_{i=1}^{24}\frac{1}{2}V_\mu^i L_i\\
  V_\mu &= \left(\begin{array}{ccccc}
    G^1_1 - \frac{2B}{\sqrt{30}} & G^1_2 & G^1_3 & \overline{X_1} & \bar{Y_1} \\
    G^2_1 & G^2_2 - \frac{2B}{\sqrt{30}} & G^2_3 & \overline{X_2} & \bar{Y_2} \\
    G^3_1 & G^3_2 & G^3_3 - \frac{2B}{\sqrt{30}} & \overline{X_3} & \bar{Y_3} \\
    X_1   & X_2 & X_3 & \frac{W^3}{\sqrt{2}} + \frac{3B}{\sqrt{30}} & W^+ \\
    Y_1   & Y_2 & Y_3 & W^- &    -\frac{W^3}{\sqrt{2}} + \frac{3B}{\sqrt{30}} 
      \end{array}
  \right)
\end{align}
ただし, $SU(3)$ゲージ場の部分は$G^i\,(i=1,\cdots,8)$をグルーオン場として
\begin{align}
  G^1_2 = G^1,\quad G^2_1 = G^2,\quad G^3_1 = G^5,\quad G^2_3 = G^6,\quad G^3_2 = G_7\nonumber\\
  G^1_1 = \frac{G^3}{\sqrt{2}} + \frac{G^8}{\sqrt{6}},\quad G^2_2=-\frac{G^3}{\sqrt{2}}+\frac{G^8}{\sqrt{6}},\quad G^3_3 = -\frac{2G^8}{\sqrt{6}}\nonumber
\end{align}
であり, $SU(2)$ゲージ場は
\begin{align}
  W^\pm = \frac{W^1\mp iW^2}{\sqrt{2}}\nonumber
\end{align}
である.
ここで, $SU(5)$ゲージ粒子として新しく$X, Y$ボゾンが導入される.
これらは
\begin{align}
  X, Y = \left(\overline{3},2, \frac{5}{3}\right)\label{XY_bosons}
\end{align}
という量子数をもつ.

ここまででゲージ場を導入できたので, $SU(5)$ゲージ群のラグランジアンを書くことができる.
はじめに共変微分を考える.
$n$表現, $\overline{n}$表現の共変微分は
\begin{align}
  (D_\mu \psi)^a = \left[\partial_\mu \delta^a_b - \frac{ig}{\sqrt{2}}(V_\mu)^a_b\right]\psi^b\nonumber\\
  (D_\mu \chi)_a = \left[\partial_\mu \delta^b_a + \frac{ig}{\sqrt{2}}(V_\mu)^b_a\right]\chi^b\nonumber
\end{align}
と表せる.
また, $\frac{n(n-1)}{2}$表現の場合
\begin{align}
  (D_\mu \psi)^{ab} = \partial_\mu \psi^{ab} - \frac{ig}{\sqrt{2}}(V_\mu)^a_c\psi^{cb}- \frac{ig}{\sqrt{2}}(V_\mu)^b_d\psi^{ad}\nonumber
\end{align}
となる.
場の強さを
\begin{align}
  (F_{\mu\nu})^a_b = \partial_\mu(A_\nu)^a_b -\frac{ig}{\sqrt{2}}(A_\mu)^a_c (A_\nu)^c_b - (\mu\leftrightarrow\nu)
\end{align}
とすると, ゲージ場の運動項は
\begin{align}
  \mathcal{L}_K = -\frac{1}{4}\mathrm{Tr}(F^{\mu\nu}F_{\mu\nu})=-\frac{1}{4}(F^{\mu\nu})^a_b(F_{\mu\nu})
\end{align}
となる.
また$\psi_{\overline{5}}$について
\begin{align}
  \mathcal{L}_{5K} &= i\overline{\psi_{\overline{5}}}\gamma^\mu D_\mu\psi_{\bar{5}}\nonumber\\
                   &= i\overline{\psi_{\overline{5}}}_a\gamma^\mu \left(\delta^a_b \partial_\mu + \frac{ig}{\sqrt{2}}(V_\mu)^a_b\right)\psi_{\bar{5}}^b\nonumber
\end{align}
さらに$\psi_{10}$について
\begin{align}
  \mathcal{L}_{10K} &= i\overline{\psi_{10}}_{ab}\gamma^\mu \left(D_\mu\psi_{10}\right)^{ab}\nonumber\\
                    &= i\overline{\psi_{10}}_{ab}\gamma^\mu \left(\delta^a_c \partial_\mu + \frac{2ig}{\sqrt{2}}(V_\mu)^a_c\right)\psi_{10}^{cb}\nonumber
\end{align}
となる.

\section{対称性の破れ}
$SU(5)$模型は$X, Y$ゲージボゾンの存在を予言する.
これらの粒子はクォークととレプトンの相互作用を引き起こすため, 陽子崩壊が起こる.
陽子は標準模型では安定である.
現在の実験データと矛盾させないためには大統一スケールが十分大きくなければならない.\footnote{これについては次の小区分で詳細を述べる.}
したがって, 大統一スケールで対称性が
\begin{align}
  SU(5)\rightarrow SU(3)_c\times SU(2)_L\times U(1)_Y\label{SU5GSM}
\end{align}
と自発的に破れると考えられている.
標準模型ではゲージボゾンは12個存在していた.
$SU(5)$模型にある24個のゲージボゾンのうち, 標準模型には現れない残りの12個の粒子が質量をもつ必要がある.
そのため少なくとも12個の南部・ゴールドストンボゾンが存在しなければならない.
また, 自発的対称性の破れは, 式(\ref{SU5GSM})のようにゲージ群を単純群から直積の群へ破るものである.
このときに群のランクを保存するものでなければならない.
したがって, そのような対称性の破れを行うものは随伴表現である.
最小模型では$24$表現ヒッグス$\phi_{\bm{24}}$を用いる.
ゲージ粒子と同じように, 
\begin{align}
  (\phi_{\bm{24}})_a^b \equiv \sum_{i=1}^{24}\frac{1}{\sqrt{2}}\phi_{\bm{24}}^i L_i
\end{align}
となる.
ゲージ場との結合は
\begin{align}
\mathcal{L}_K = \frac{1}{2}\sum_{i=1}^{24}[D_\mu\phi_{\bm{24}}]^\dagger [D^\mu\phi_{\bm{24}}]\nonumber
\end{align}
となる.
共変微分を具体的に求める.
24表現は$\bm{24}\oplus \bm{1} = \bm{5}\otimes \overline{\bm{5}}$であるため,
\begin{align}
  \phi_{\bm{24}b}^{\,\,a} = \psi_{\bm{5}}^a \psi_{\bm{\overline{5}}b} -\frac{1}{5}\delta^a_b \psi^c_{\bm{5}}\psi_{\bm{\bar{5}}c}
\end{align}
と構成されている.
それぞれの表現の変換は
\begin{align}
  \psi_{\bm{5}}^a \rightarrow \psi^a_{\bm{5}}+\delta \psi_{\bm{5}}^a = \psi_{\bm{5}}^a -i\frac{g}{2}\varepsilon^i L^a_{ic}\psi_{\bm{5}}^c \nonumber\\
  \psi_{\bm{\overline{5}}a} \rightarrow \psi_{\bm{\bar{5}}a}+\delta \psi_{\bm{5}}^a = \psi_{\bm{5}}^a +i\frac{g}{2}\varepsilon^i L^c_{ia}\psi_{\bm{5}c}\nonumber
\end{align}
となるから, $\phi_{\bm{24}}$は
\begin{align}
  \phi_{\bm{24}b}^a \rightarrow \phi_{\bm{24}b}^a -i\frac{g}{2}[L_i,\phi_{\bm{24}}]^a_b
\end{align}
と変換を受ける.
したがって共変微分は
\begin{align}
  (D_\mu\phi_{\bm{24}})^a_b = \partial_\mu \phi_{\bm{24}a}^{\,\,b}  -\frac{ig}{\sqrt{2}}[V_\mu, \phi_{\bm{24}}]^b_{a}
\end{align}
となる.
ヒッグスの運動項は
\begin{align}
  \mathcal{L}_K = \mathrm{Tr}[(D_\mu \phi_{\bm{24}})^\dagger (D^\mu \phi_{\bm{24}})]
\end{align}
とまとめることができる.
ゲージボゾンの質量は24表現ヒッグスの真空期待値を$\Sigma = \langle 0|\phi_{\bm{24}}| 0\rangle$としたとき
\begin{align}
  \mathcal{L}_M &= \mathrm{Tr}[(D_\mu \phi_{\bm{24}})^\dagger (D^\mu \phi_{\bm{24}})]|_{\phi_{\bm{24}}=\Sigma}\nonumber\\
                &= \frac{g^2}{2}\mathrm{Tr}\left([V_\mu, \Sigma]^\dagger [V^\mu,\Sigma]\right) = m_{ij}^2 V_\mu^{i\dagger}V^{\mu j}
\end{align}
これによりポテンシャルは
\begin{align}
  V(\phi_{\bm{24}})) = -\frac{\mu^2}{2}\mathrm{Tr}(\phi_{\bm{24}}^2) +\frac{1}{4}a(\mathrm{Tr}(\phi_{\bm{24}}^2))^2 + \frac{1}{2}b\mathrm{Tr}(\phi_{\bm{24}}^4) + \frac{1}{3}c\mathrm{Tr}\phi_{\bm{24}}^3\nonumber
\end{align}
となる.
% NOTE: 下の文章は一般的な対称性の破れで考えられているものだが, 既に具体的なので省いた.
% これは$\phi_{\bm{24}}\rightarrow U\phi_{\bm{24}} U^\dagger$と変換する.
簡単のために$\mathbb{Z}_2$対称性を課すと, $c=0$となる.
真空期待値は次のようなとり方が考えられる.
\begin{align}
  \Sigma &= v\,\mathrm{diag}(1,1,1,1,-4),\quad(b<0)\label{vev1}\\
  \Sigma &= v\,\mathrm{diag}\left(1,1,1,-\frac{3}{2},-\frac{3}{2}\right),\quad(b>0)\label{vev2}
\end{align}
式(\ref{vev1})の場合, ゲージ対称性は$SU(4)\times U(1)$に破れ, 式(\ref{vev2})の場合は$SU(3)_c\times SU(2)_L\times U(1)_Y$に破れる.
したがって, 後者の場合を考える.
この場合のポテンシャルの最小値は$b>0,\, a>-{7b}/{15}$の条件のもとで
\begin{align}
  \left.\frac{\partial V(\phi_{\bm{24}})}{\partial \phi_{\bm{24}}} \right|_{\phi_{\bm{24}}=\Sigma} = 0
\end{align}
より求められ, 
\begin{align}
  v^2 = \frac{2\mu^2}{15a+7b}\nonumber
\end{align}
となる.
ゲージボゾンの質量は24表現ヒッグスに真空期待値をもたせると
\begin{align}
  [V_\mu, \Sigma] \sim \left(\begin{array}{@{}c|c@{}}
      {\Large{\mbox{$0$}}} &
      \begin{matrix}
        -\frac{5}{2}\overline{X_1} & -\frac{5}{2}\bar{Y_1} &   \\
        -\frac{5}{2}\overline{X_2} & -\frac{5}{2}\bar{Y_2} &   \\
        -\frac{5}{2}\overline{X_3} & -\frac{5}{2}\bar{Y_3} &   \\
      \end{matrix} \\
      \hline
      \begin{matrix}
        \frac{5}{2}X_1 & \frac{5}{2}X_2 & \frac{5}{2}X_3 \\
        \frac{5}{2}Y_1 & \frac{5}{2}Y_2 & \frac{5}{2}Y_3 \\
      \end{matrix} & {\large{\mbox{$0$}}}
  \end{array}\right)
\end{align}  
となる.
これにより$X, Y$ボゾンの質量は
\begin{align}
  m_X^2 = m_Y^2 = \frac{25}{8}g^2v^2
\end{align}
となる.

標準模型で表れるヒッグス粒子$\Psi$は5表現に属する.
\begin{align}
  \phi_{\bm{5}} = \left(
  \begin{array}{c}
      T^1 \\
      T^2 \\
      T^3 \\
      \phi^+ \\
      -\phi^0
  \end{array}\right) = T\left(\bm{3}, 1,-\frac{1}{3}\right)\oplus H_2\left(1,\bm{2},\frac{1}{2}\right)\label{Higgs_5}
\end{align}
真空期待値は
\begin{align}
  \langle\phi_{\bm{5}}\rangle = \frac{1}{\sqrt{2}}\left(
  \begin{array}{c}
      0 \\
      0 \\
      0 \\
      0 \\
      v_0
  \end{array}\right) = \frac{1}{\sqrt{2}}v_0\delta_5^a,\quad(v_0\sim 246\,[\mathrm{GeV}]) \label{Higgs_VEV_5}
\end{align}
のようにして, 標準模型と同じ値を取る.
もし24表現ヒッグスと5表現ヒッグスに交わりがない場合, ポテンシャルは
\begin{align}
  V(\phi_{\bm{5}}) = -\frac{{\mu_5}^2}{2}\phi_{\bm{5}}^\dagger \phi_{\bm{5}} + \frac{\lambda}{4}(\phi_{\bm{5}}^\dagger \phi_{\bm{5}})^2
\end{align}
である.\footnote{第2章のポテンシャルの規格化とは異なる}
%NOTE: Notationが微妙にちがう. BEGNに合わせると \frac{\mu^2}{2}になるので注意.
このとき, $v_0^2 =\frac{2\mu_5^2}{\lambda}$となり, 標準模型に表れるゲージボゾンは
\begin{align}
  m_W^2 = \frac{g^2 v_0^2}{4}
\end{align}
と質量を持つことになる.
\subsection{階層性問題}
ここまでで標準模型よりも高い対称性を持つ理論を構築した場合, どのように対称性を破り, 標準模型の対称性を実現するのかをみた.
ところが, 大統一理論を考えた場合, 一般的にエネルギースケールの階層性が問題となる.
ここでは大統一理論に存在する階層性問題について触れる.
% 1st version: Based on 素粒子の標準模型を超えて

素粒子標準模型は電弱スケールである$M_W\sim100\,[\mathrm{GeV}]$まで高エネルギー加速器実験結果を説明することができる.
一方で標準模型を超えた物理(Beyond the Standard Model; BSM)が加速器実験で検証されるには電弱スケールよりも高いエネルギーにより, その実験を検証することが可能となる.

この見方を変えると, 現在の標準模型はこのようなBSMの有効理論であると考えることができる.
したがって素粒子標準模型の理論の適用範囲は何らかのエネルギースケールである$\Lambda$まで有効であり, $\Lambda$以上のエネルギーでは別の理論へ移り変わると考えられている.

大統一理論や重力が含まれる理論では, このカットオフは$\Lambda\sim M_{\mathrm{GUT}}$や$\Lambda\sim M_{\mathrm{pl}}$程度であるとそれぞれ考えられており, $M_W$に比べて13桁程度の乖離が存在する.

標準模型に登場する粒子はヒッグス粒子の真空期待値に比例するため, これらは電弱スケールに質量が存在することとなる.
これらはゲージ理論により説明されるが, ヒッグス粒子の質量を説明できる主導原理は標準模型に存在しない.
標準模型に表れるヒッグス粒子の質量を$m_h$とした場合, いかにして$m_h \ll \Lambda$を保つかが大きな問題となっている.

\subsection{Doublet-triplet splitting problem}
ここでは$SU(5)$大統一理論を考える.
$SU(5)$大統一理論では, 5表現ヒッグスと24表現ヒッグスを考えることができた.
それぞれ$H_2$,\,$\phi_{\bm{24}}$とおく.
これらのヒッグス粒子によるポテンシャルを考える.
$\mathbb{Z}_2$対称性を課すと,
\begin{align}
  V(H,\,\Phi) = -\frac{1}{2}\nu^2 H_2^\dagger H_2 + \frac{\lambda}{4}(H_2^\dagger H_2)^2 + H_2^\dagger[\alpha \mathrm{Tr}(\phi_{\bm{24}}^2)+\beta(\Phi_{\bm{24}}^2)]H_2 \label{eq5-1} 
\end{align}
となる.
ここで, $\phi_{\bm{24}}$の最小化は式(\ref{eq5-1})の第3項の内部のみで行われていると考える.
これは式(\ref{eq5-1})は階層性のもとでは, 多項式全体の最小化の影響よりも, 十分影響を与えるためである.

ただし, このように真空期待値を取った場合, $Y$ボゾンに質量を与えうる$H_2^\alpha$と$\phi_{\bm{24}5}^\alpha$という2つのカラー三重項ヒッグス場が存在したとしても片方のヒッグス場のみ質量を与え, もう一方は質量がないままとなる.

%EOF

\section{フェルミオン質量}
フェルミオンの質量項は$\overline{\psi}_R\chi_L +\mathrm{h.c.}$で書かれる.
カイラリティを顕にすると
\begin{align}
  \psi_L C \chi_L + \mathrm{h.c.} = \chi_L^T C \psi_L + \mathrm{h.c.} \equiv \overline{\psi_R^c}\chi_L
\end{align}
と書かれる. $C$は荷電共役変換の演算子である.
このとき, 直積表現は
\begin{align}
  \overline{\bm{5}}\otimes \overline{\bm{5}} &= \bar{\bm{10}}\oplus\bar{\bm{15}}\label{decon_55}\\
  \overline{\bm{5}}\otimes {\bm{10}} &= \bm{5}\oplus\bm{45}\label{decon_510}\\
  \bm{10}\otimes \bm{10} &= \overline{\bm{5}}\oplus\bar{\bm{45}}\oplus\bm{50}\label{decon_1010}
\end{align}
と分解されるため, $\bm{5}$表現か$\bm{45}$表現に属するスカラー場とのみ結合することがわかる.
既に見たように, (\ref{Higgs_5})のように標準模型で現れるヒッグス粒子を考えることができた.
そのため標準模型で現れる質量の関係を得るためには$\bm{5}$表現ヒッグスを用いることで記述することができる.
また, $\mathrm{SU(5)}$対称性を破る随伴表現ヒッグスはこれらのフェルミオンと結合しないため, 大統一スケールとフェルミオンの質量スケールを離して考えることが自然と可能となる.

はじめに$\overline{\bm{5}}\otimes {\bm{10}}$の場合を考える.
\begin{align}
  \mathcal{L}_Y \supset Y_5^{mn} (\psi^T_{\bm{\overline{5}}mL})_a C (\psi_{\bm{10}nL})^{ab}(\phi_{\bm{5}})_b^\dagger +\mathrm{h.c.}\label{yukawa_5_10}
\end{align}
ここで, $Y_5$は世代ごとの湯川結合を行列で表したものであり, $\{m, n= 1,2,3\}$である.
ヒッグスに真空期待値を式(\ref{Higgs_VEV_5})のように持たせると,
\begin{align}
  Y_5^{mn} (\psi^T_{\bm{\overline{5}}mL})_a C (\psi_{\bm{10}nL})^{ab}(\phi_{\bm{5}})_b^\dagger +\mathrm{h.c.} \rightarrow\quad -\frac{v_0}{2}Y_5^{mn}\left(\overline{d_{mR}}d_{nL} + \overline{e_{mR}^+}e^+_{nL}\right) + \mathrm{h.c.}
\end{align}
となり, 下系列クォークと荷電レプトンの質量を導くことができ, 
\begin{align}
  \frac{v_0}{\sqrt{2}}Y_5^{mn} &= \mathrm{diag}(m_d, m_s, m_b)\quad (= M_d)\nonumber\\
                               &= \mathrm{diag}(m_e, m_\mu, m_\tau)\quad (= M_e)
\end{align}
となる.
一方で,$ \bm{10}\otimes \bm{10}$の場合は
\begin{align}
  \mathcal{L}_Y \supset Y_{10}^{mn} \varepsilon_{abcde} (\psi^{T}_{\bm{\overline{5}}mL})^{ab}C (\psi_{\bm{10}nL})^{cd} \phi_{\bm{5}}^e \label{u_Yukawa}
\end{align}
となる.
ここで$\varepsilon_{abcde}=1$を満たす完全反対称テンソルである.
上系列クォークの質量は$d=4$の時に得ることができる.
真空期待値を式(\ref{Higgs_VEV_5})のようにとると, 式(\ref{u_Yukawa})は
\begin{align}
  Y_{10}^{mn} \varepsilon_{abcde} (\psi^{T}_{\bm{\overline{5}}mL})^{ab}C (\psi_{\bm{10}nL})^{cd} \phi_{\bm{5}}^e \rightarrow&\quad 4Y_{10}^{mn} \varepsilon_{abc45} (\psi^{T}_{\bm{\overline{5}}mL})^{ab}C (\psi_{\bm{10}nL})^{cd}\frac{v_0}{\sqrt{2}} + \mathrm{h.c.} \nonumber\\
                                                                                                                            &= 4Y_{10}^{mn} \varepsilon_{\alpha \beta \gamma} (\psi^{T}_{\bm{\overline{5}}mL})^{\alpha\beta}C (\psi_{\bm{10}nL})^{\gamma 4}\frac{v_0}{\sqrt{2}} + \mathrm{h.c.}\nonumber\\
                                                                                                                            &= \frac{4}{\sqrt{2}}v_0 Y_{10}^{mn}\overline{u_{mR}}u_{nL} + \mathrm{h.c.}
\end{align}
と表すことができる.
ここで述べた質量はいずれも大統一のエネルギースケールで成立するものである.
そのため電弱スケールでの実験値と比較するためには繰り込み群方程式を解くことによってエネルギー依存性を含めて考える必要がある.
\section{SU(5)大統一理論による予言と問題点}
$SU(5)$大統一理論では, 素粒子標準模型に現れる粒子を$SU(5)$群の$\overline{\bm{5}}$表現や$\bm{10}$表現, 随伴表現の$\bm{24}$表現に当てはめて考えた.
これにより素粒子標準模型には存在しない現象が予言される.
\subsection{陽子崩壊}
$SU(5)$大統一理論では式(\ref{XY_bosons})で見たように, 新たに$X, Y$ゲージボゾンの存在が予言される.
ゲージ相互作用は次のような形をとる.
\begin{align}
  \mathcal{L}_{Int} &= -\frac{g_5}{\sqrt{2}}\left[ \overline{X}_\mu^i(\overline{d}_{iR}\gamma^\mu e^c_R + \overline{d}_{iL}\gamma^\mu e^c_L -\varepsilon_{ijk} \overline{u}_L^{c\,j}\gamma^\mu u_L^k) \right.\nonumber\\ &\qquad\qquad+\left.\overline{Y}_\mu^i(-\overline{d}_{iR}\gamma^\mu \nu_R^c -\overline{u}_{iL}\gamma^\mu e_L^c + \varepsilon_{ijk}\overline{u}_L^{c\,k}\gamma^\mu d_L^j)\right] \label{PD_L}
\end{align}
これらの項により, 標準模型では存在しないクォークとレプトンの相互作用が起こる.
図\ref{fig_X_boson}は$X, Y$ボゾンによる相互作用をファインマン図を用いて表したものである.
\begin{figure}[ht]
  \begin{center}
    \begin{tabular}{ccc}
      \begin{minipage}[b]{0.20\columnwidth}
        \centering
        \begin{tikzpicture}
          \begin{feynhand}
            \vertex [particle] (e) at (-0.8,1.26) {$e^c$};
            \vertex [particle] (d) at (-0.8,-1.26) {$\overline{d}$};
            \vertex (x_in) at (0,0);
            \vertex (x_out) at (1.6,0);
            \propag [fermion] (x_in) to (e);
            \propag [fermion] (d) to (x_in);
            \propag [charged boson] (x_out) to [edge label =$X$] (x_in);
          \end{feynhand}
        \end{tikzpicture}
      \end{minipage} &
      \begin{minipage}[b]{0.20\columnwidth}
        \centering
        \begin{tikzpicture}
          \begin{feynhand}
            \vertex [particle] (e) at (-0.8,1.26) {$e^c$};
            \vertex [particle] (d) at (-0.8,-1.26) {$u$};
            \vertex (x_in) at (0,0);
            \vertex (x_out) at (1.6,0);
            \propag [fermion] (x_in) to (e);
            \propag [fermion] (d) to (x_in);
            \propag [charged boson] (x_out) to [edge label =$Y$] (x_in);
          \end{feynhand}
        \end{tikzpicture}
      \end{minipage} &
      \begin{minipage}[b]{0.20\columnwidth}
        \centering
        \begin{tikzpicture}
          \begin{feynhand}
            \vertex [particle] (e) at (-0.8,1.26) {$\nu^c$};
            \vertex [particle] (d) at (-0.8,-1.26) {$d$};
            \vertex (x_in) at (0,0);
            \vertex (x_out) at (1.6,0);
            \propag [fermion] (x_in) to (e);
            \propag [fermion] (d) to (x_in);
            \propag [charged boson] (x_out) to [edge label =$Y$] (x_in);
          \end{feynhand}
        \end{tikzpicture}
      \end{minipage}
    \end{tabular}
  \end{center}
  \caption{$X$ボゾン, $Y$ボゾンによるクォークとレプトンの相互作用.}
  \label{fig_X_boson}
\end{figure}
\begin{figure}[ht]
  \centering
  \begin{tikzpicture}
    \begin{feynhand}
      \vertex [particle] (e) at (-0.8,1.26) {$e^c$};
      \vertex [particle] (d) at (-0.8,-1.26) {$d$};
      \vertex (x_in) at (0,0);
      \vertex (x_out) at (1.6,0);
      \vertex [particle] (u) at (2.4,1.26) {$u^c$};
      \vertex [particle] (uc) at (2.4,-1.26) {$u$};
      \vertex (free) at (2.0,0);
      \vertex [particle] (fu_in) at (2.8,-1.26) {$u$};
      \vertex [particle] (fu_out) at (2.8,1.26) {$u$};
      \propag [fermion] (x_in) to (e);
      \propag [fermion] (d) to (x_in);
      \propag [fermion] (u) to (x_out);
      \propag [fermion] (x_out) to (uc);
      \propag [fermion] (fu_in) to (free);
      \propag [fermion] (free) to (fu_out);
      \propag [charged boson] (x_out) to [edge label =$X$] (x_in);
    \end{feynhand}
  \end{tikzpicture}
  \caption{$X$ボゾンを介した陽子崩壊のファインマン図}
  \label{fig_X_proton_decay}
\end{figure}

ここで陽子の崩壊率を導く.
まず, ゲージボゾン$X$の質量項は式(\ref{PD_L})から
\begin{align}
  \mathcal{M_X} = -\frac{M_X^2}{2}(\overline{X_\mu})^\alpha_a (X^\mu)^a_\alpha \label{eqofX}
\end{align}
となる.
式(\ref{eqofX})にて導入した$(\overline{X_\mu})^\alpha_a $と$(X^\mu)^a_\alpha$はそれぞれ
\begin{align}
  (\overline{X_\mu})^\alpha_a &= \frac{1}{\sqrt{2}}\frac{g_5}{M_X^2}\left[\overline{(d^c)}^\alpha \gamma_\mu \epsilon_{ab} L^d -\overline{e^c}\epsilon_{ba}\gamma_\mu q^{\alpha b} + \overline{q}_{\beta a}\gamma_\mu \epsilon^{\alpha\beta\gamma} u_\gamma^c   \right] \nonumber\\
  (X^\mu)^a_\alpha  &=\frac{1}{\sqrt{2}} \frac{g_5}{M_X^2} \left[ (\overline{L})_b \epsilon^{ba}\gamma_\mu d_\alpha^c -\overline{q}_{b\alpha} \gamma_\mu \epsilon^{ab}e^c + \epsilon_{\gamma \beta \alpha} (\overline{u}^c)^\gamma \gamma_\mu q^{a\beta}  \right] \nonumber
\end{align}
となる.
これらを式(\ref{PD_L})に代入し, 次元6の陽子崩壊を導くラグランジアンは
\begin{align}
  \mathcal{L}_{d=6} = \frac{g_5^2}{M_X^2} \epsilon_{\alpha\beta\gamma} (\overline{u^c})^\alpha \gamma_\mu q^{a\beta}\{\overline{e^c}\epsilon_{ab}\gamma^\mu q^{\gamma b} + (\overline{d^c})^\gamma \gamma^\mu \epsilon_{ab} L^b \} + \mathrm{h.c.}
\end{align}
となる.

陽子崩壊の崩壊パターンの可能性は様々なものがあるが, 図\ref{fig_X_proton_decay}で示されている
\begin{align}
  p \quad \longrightarrow\quad \pi^0 \quad + \quad e^c\nonumber
\end{align}
の例を取り上げる.

これらの崩壊を引き起こす演算子は
\begin{align}
  O^{I} &= \frac{g_5^2}{M_X^2} \epsilon_{\alpha\beta\gamma} (\overline{u^c})^\alpha \gamma_\mu u^\beta \overline{e^c}\gamma^\mu d^\gamma,\nonumber\\
  O^{II} &= \frac{g_5^2}{M_X^2} \epsilon_{\alpha\beta\gamma} (\overline{u^c})^\alpha \gamma_\mu u^\beta  \gamma^\mu d\nonumber
\end{align}
の2つを考えることができる. Nathらにより\cite{nathProtonStabilityGrand2007}これらの演算子による崩壊率は計算されているが, おおよそ
\begin{align}
  \Gamma(p\rightarrow \pi^0 e^+) \sim \alpha_5 \frac{m_p^5}{M_X^4}\label{decayrate}
\end{align}
である.
このことから$X,Y$ボゾンによる陽子の崩壊率はそれぞれのボゾンの質量の4乗の項によりおおよそ決められる.

典型的な大統一スケールや結合定数の値を考えると
\begin{align}
  \tau(p\rightarrow \pi^0 e^+) \sim 4\times 10^{29} \,[\mathrm{yr}] \label{plife_th}
\end{align}
となる.
一方で, スーパーカミオカンデでは陽子崩壊の候補となるようなシグナルは発見されていないため, 陽子の寿命の下限が定められる\cite{collaborationSearchProtonDecay2020}.
\begin{align}
  \tau(p\rightarrow \pi^0 e^+) \sim 2.4\times10^{34}\,[\mathrm{yr}]\nonumber
\end{align}
これは理論的に導かれた式(\ref{plife_th})の寿命の予想を上回るため, $SU(5)$大統一理論は観測の制限と整合性が取れないことがわかる.
\begin{comment}
\subsection{電荷の量子化}
式(\ref{quantum_Q})で述べているが, 標準模型では予言されない電荷の量子化が行える.
これは$SU(5)$大統一理論がクォーク, レプトンを $SU(5)$の単純群で記述し, 式(\ref{GUT-5rep})のように$\overline{\bm{5}}$表現に当てはめることで
\begin{align}
  \mathrm{Tr}Q = 0 \nonumber
\end{align}
と標準模型では与えられない条件を満足させているためである.
\end{comment}
\subsection{ワインバーグ角}
標準模型に現れるフェルミオンやゲージ粒子が, $\mathrm{SU(5)}$大統一理論でどのように当てはめられるのかを見てきた.
ゲージ結合定数が大統一スケールで$g_5$と統一されたことにより, 理論的にワインバーグ角を決定することができる.
それを見るために$\mathrm{U(1)_Y}$ゲージ群の結合定数を$SU(5)$の中で規格化する.
フォトン場を$A_\mu$, 中性弱ボゾンを$Z_\mu$とすると
\begin{align}
  D_\mu &\supset \partial_\mu -i\frac{g_5}{2}[W^0 L_{11} + B L_{12}]\nonumber\\
        &\quad=\partial_\mu -i\frac{g_5}{2}[(\sin\theta_W L_{11} + \cos\theta_W L_{12})A_\mu + (\cos\theta_W L_{11} -\sin\theta_W L_{12})Z_\mu]\nonumber\\
        &\quad\equiv \partial_\mu -i[eQA_\mu + g' Q_Z Z_\mu]\nonumber
\end{align}
となる.
このとき, 電荷を表す演算子$Q$は,
\begin{align}
  eQ = \frac{g_5}{2}(\sin\theta_W L_{11} + \cos\theta_W L_{12})\nonumber
\end{align}
と関係する.
一方で電荷演算子はアイソスピン$I_3 = L_{11}$と超電荷$Y=\sqrt{\frac{5}{3}}L_{12}$を用いると
\begin{align}
  Q = L_{11} + \sqrt{\frac{5}{3}}L_{12}\nonumber
\end{align}
とかける.
これらを比較することで,
\begin{align}
  e &= g_5\sin\theta_W\nonumber\\
  g_Y&= \frac{e}{\cos\theta_W}=\sqrt{\frac{3}{5}}g_5 = g_5 \tan\theta_W\label{re_gY}\\
  \sin\theta_W &=\sqrt{\frac{3}{8}}\nonumber
\end{align}
と大統一スケールでの値を予言することができる.
$\sin\theta_W$は標準模型では値を予言することはできなかったが, 大統一理論では$\sin\theta_W$を予言ができることが示された.

これらの予想は$SU(5)$ゲージ群において成り立つものである.
そのため大統一スケールよりも低いエネルギースケールでどのような値を取るかはくりこみ群方程式を解くことによって導く必要がある.
量子補正を1次まで含めた式(\ref{bfnc_1loop})は, ワインバーグ角と結合定数の規格化を行った式(\ref{re_gY})を含めると
\begin{align}
  \frac{1}{\alpha_3(\mu)} &= \frac{1}{\alpha_{\mathrm{GUT}}} +\frac{1}{2\pi}\left(11-\frac{4}{3}n_f\right)\mathrm{ln}\left(\frac{\mu}{M_{\mathrm{GUT}}}\right)\nonumber\\
  \frac{1}{\alpha_2(\mu)} &= \frac{\sin^2\theta_W}{\alpha_{Y}} =  \frac{1}{\alpha_{\mathrm{GUT}}} +\frac{1}{2\pi}\left(22-\frac{4}{3}n_f - \frac{n_H}{6}\right)\mathrm{ln}\left(\frac{\mu}{M_{\mathrm{GUT}}}\right)\nonumber\\
  \frac{1}{\alpha_1(\mu)} &= \frac{3}{5}\frac{\cos^2\theta_{W}}{\alpha_{Y}} =\frac{1}{\alpha_{\mathrm{GUT}}} -\frac{1}{2\pi}\left(\frac{4}{3}n_f+\frac{1}{10}n_f\right)\mathrm{ln}\left(\frac{\mu}{M_{\mathrm{GUT}}}\right)\nonumber
\end{align}
と表すことができる.
ただし, $n_f, n_H$はそれぞれ世代数とヒッグス二重項の数である.

これを$\sin^2\theta_W$について解く.
$n_f =3, n_H = 1$ として
\begin{align}
  \sin^2{\theta_W} &= \frac{3}{8}\left( 1 - \frac{109}{18\pi}\alpha_{Y}(\mu)\mathrm{ln}\left(\frac{M_{\mathrm{GUT}}}{\mu}\right) \right)\nonumber\\
  \frac{\alpha_{Y}{(\mu)}}{\alpha_{3}{(\mu)}} &= \frac{3}{8}\left( 1 - \frac{67}{6\pi} \mathrm{ln}\left(\frac{M_{\mathrm{GUT}}}{\mu}\right)\right)\label{alpha_em}
\end{align}
となる.

式(\ref{alpha_em})について$\mu=M_Z$の場合を考え, $\alpha$の値を決めると
\begin{align}
  \alpha_{GUT} = 0.0244\pm 0.0002\nonumber\\
  \sin^2 \theta_W = 0.214\pm 0.004\nonumber
\end{align}
と導かれる. \cite{burasAspectsGrandUnification1978}
一方で, 最新の実験値は
\begin{align}
  \sin^2\theta_W = 0.23129\pm0.00004
\end{align}
と与えられるため, $SU(5)$大統一理論が予言するワインバーグ角は電弱スケールでの測定値と合わない.

\subsection{フェルミオン質量の関係}
$SU(5)$大統一模型では, 標準模型に存在するヒッグス粒子を$\bm{5}$表現に当てはめて考えた.
湯川結合は式(\ref{yukawa_5_10}と式(\ref{u_Yukawa})で書くことができ, 前者の下系列クォークと荷電レプトンについては, 
\begin{align}
  M^d = M^e = \frac{v_0}{\sqrt{2}}Y_5^{mn}
\end{align}
となる.
このことから
\begin{align}
  m_d = m_e,\quad m_s = m_\mu,\quad m_b = m_\tau \label{Same_mass}
\end{align}
と下系列クォークと荷電レプトンの質量が大統一スケールでは等しくなることが示されている.
湯川結合定数もくりこみ群方程式によりエネルギー依存性を導けるため, これを用いて低エネルギースケールでどのような予言がされるのか考える必要がある.
これによると\cite{burasAspectsGrandUnification1978,chanowitzPriceNaturalFlavour1977}
\begin{align}
  \mathrm{ln}\left[ \frac{m_d(\mu_1)}{m_e(\mu_1)}\right]=\mathrm{ln}\left[ \frac{m_d(\mu_2)}{m_e(\mu_2)}\right] + \frac{4}{11-\frac{2n_q}{3}}\mathrm{ln}\left[ \frac{\alpha_3(\mu_1)}{\alpha_5(\mu_2)}\right]+\frac{3}{2n_q}\mathrm{ln}\left[\frac{\alpha_1(\mu)}{\alpha_5(\mu_2)}\right]\label{rel_mass}
\end{align}
と質量比を導くことができる.
ここで$n_q$はクォークの質量$m$が$m<\mu_1$を満たすものの数である.
式(\ref{rel_mass})は他の世代のフェルミオンに対しても導くことができる.
これまでの実験でクォークは6種類は存在することが明かされている.
$m_\tau$の測定値を代入することによって
\begin{align}
  m_b = 5-5.5\,[\mathrm{GeV}],\quad (50\,[\mathrm{MeV}]\leq \Lambda_{\overline{MS}} \leq 500\,[\mathrm{MeV}])
\end{align}
となる.
さらに高次の影響を含めた計算を行うと
\begin{align}
  m_b = 5.3\,[\mathrm{GeV}]
\end{align}
と見積もられ, 実験値である$m_b = 4.9\,[\mathrm{GeV}]$と等しいオーダーとなる.
また, $n_q$による依存性があり, もし3世代よりも大きな世代構造があるとすれば, 実験値から離れることが明らかにされている\cite{nanopoulosLimitsNumberFlavours1979,burasAspectsGrandUnification1978}.

一方で, 先程述べているように式(\ref{rel_mass})はすべての世代で成り立つ.
同じように$m_s$を導くと
\begin{align}
  m_s = 500\,[\mathrm{MeV}]
\end{align}
となる.
ところが実験では$m_s = 93.5\,[\mathrm{MeV}]$であるため, オーダーも異なってしまう.

さらに, 式(\ref{Same_mass})から
\begin{align}
  \frac{m_d}{m_s} = \frac{m_e}{m_\mu} \label{rel-mass-e}
\end{align}
となる.
ここで, 測定から$\frac{m_e}{m_\mu}=\frac{1}{207}$であることが明らかにされているが, 一方で異なる実験から$\frac{m_d}{m_s} = \frac{1}{24}$となるため, この関係は式(\ref{rel-mass-e})を満足しない\cite{dominguezDeterminationQuarkMass1979,dominguezZeroMassQuarksU11978,langackerLightquarkMassSpectrum1979,langackerQuarkMassDifferences1979,weinbergFestschriftIIRabi1977}.
これらはクォークの持つ質量が小さく, 非摂動効果が大きいためであると考えられるが, 詳細は十分に明らかにされていない.

%EOF
