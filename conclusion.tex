% ---------------------------------------
%
%  Conclusion
%  conclusion.tex
%  Program modified by Yasutoki Takamura
%  Last Modified Jan 27 2025
%
% ---------------------------------------
本論文では素粒子標準模型がどのような理論体型であったのか確認し, さらに標準模型では説明できない物理現象を簡単にまとめた.
その中でも場の理論による繰り込みという手法を用いることで,  ゲージ結合定数の大きさはエネルギー依存性をもち, 3つのゲージ結合定数の大きさは高いエネルギーになるにつれて近づくことがわかった.

標準模型ではゲージ対称性によって相互作用が記述される.
大統一理論はそのゲージ対称性を高いものへ拡張し, 標準模型を部分的に含むことによって理論を構築するものであり, 非常に自然な拡張だと考えられる.

最小$SU(5)$大統一理論は陽子崩壊が観測されていないため実験的に排除されてしまっているが, 単純群で標準模型を含むアイデアは画期的なものであり, 拡張の余地がある.

本論文では$\bm{15}$表現ヒッグスを加えることでくりこみ群方程式の係数が実際に変化し, ゲージ結合定数が統一される様子を確認できた.
このように電弱スケールと大統一スケールの間に粒子が存在する場合, 最小$SU(5)$模型では叶わなかったゲージ結合定数の大きさを統一する可能性が高まる.
しかし, そのような中間スケールに粒子の存在する場合, 新粒子の性質によっては陽子の崩壊を非常に早めてしまう可能性もある.
本論文ではそのような解析まで行えていないが, 低スケールに$\bm{15}$表現ヒッグスのレプトクォーク成分が存在しているため, 既にこの模型も実験的に棄却され始めている\cite{collaborationSearchProtonDecay2020}.

一方で, ニュートリノが質量を持つことは実験的に揺るぎないものであり, タイプII型のシーソー機構を含むような拡張を標準模型に施すのであればスカラー三重項を理論に加える必要がある.
そのような拡張を有効演算子として加える場合であっても, 高いゲージ対称性を持つ大統一理論がそのような粒子を含む必要があるため, $SU(5)$大統一理論に対して$\bm{15}$表現ヒッグスを加えた拡張を行い研究を行うことは意義があると考える.

今後の研究では$\bm{15}$表現ヒッグスが$SU(5)$大統一理論に拡張され, 陽子崩壊の制限を回避し, ニュートリノ質量が観測事実と整合性のとれた説明ができる模型の探査を行う.
本論文で述べたように$SU(5)$大統一模型が持つ問題点はニュートリノ質量以外にもいくつか存在していた.
それらを解決できるような模型を考案することを今後の研究としたい.

