\documentclass[titlepage]{jsbook}
%DIF LATEXDIFF DIFFERENCE FILE
%DIF DEL /var/folders/lt/gf73zdxd4bb16_mzpg3kzp7r0000gn/T/ymSMY8w0OJ/latexdiff-vc-HEAD/main.tex   Fri Jul 19 00:09:03 2024
%DIF ADD main.tex                                                                                 Thu Jun 20 11:29:07 2024

% ----- preambles ------------------
\usepackage{geometry}
\usepackage{bm}
\usepackage{here}
\usepackage{amsmath,amsthm,amssymb}
\usepackage{ascmac}
\usepackage[dvipdfmx]{graphicx}
% ----------------------------------

% ----- Settings for amsthm -----
\theoremstyle{plain}
\newtheorem{thm}{Theorem}
\newtheorem*{thm*}{Theorem}

\theoremstyle{definition}
\newtheorem{dfn}{Definition}
% -------------------------------

% ----- Geometry setting ---------------------------
\geometry{left=25mm,right=25mm,top=30mm,bottom=30mm}
% ---------------------------------------------------
%DIF PREAMBLE EXTENSION ADDED BY LATEXDIFF
%DIF CFONT PREAMBLE %DIF PREAMBLE
\RequirePackage{color}\definecolor{RED}{rgb}{1,0,0}\definecolor{BLUE}{rgb}{0,0,1} %DIF PREAMBLE
\DeclareOldFontCommand{\sf}{\normalfont\sffamily}{\mathsf} %DIF PREAMBLE
\providecommand{\DIFadd}[1]{{\protect\color{blue} \sf #1}} %DIF PREAMBLE
\providecommand{\DIFdel}[1]{{\protect\color{red} \scriptsize #1}} %DIF PREAMBLE
%DIF SAFE PREAMBLE %DIF PREAMBLE
\providecommand{\DIFaddbegin}{} %DIF PREAMBLE
\providecommand{\DIFaddend}{} %DIF PREAMBLE
\providecommand{\DIFdelbegin}{} %DIF PREAMBLE
\providecommand{\DIFdelend}{} %DIF PREAMBLE
\providecommand{\DIFmodbegin}{} %DIF PREAMBLE
\providecommand{\DIFmodend}{} %DIF PREAMBLE
%DIF FLOATSAFE PREAMBLE %DIF PREAMBLE
\providecommand{\DIFaddFL}[1]{\DIFadd{#1}} %DIF PREAMBLE
\providecommand{\DIFdelFL}[1]{\DIFdel{#1}} %DIF PREAMBLE
\providecommand{\DIFaddbeginFL}{} %DIF PREAMBLE
\providecommand{\DIFaddendFL}{} %DIF PREAMBLE
\providecommand{\DIFdelbeginFL}{} %DIF PREAMBLE
\providecommand{\DIFdelendFL}{} %DIF PREAMBLE
\newcommand{\DIFscaledelfig}{0.5}
%DIF HIGHLIGHTGRAPHICS PREAMBLE %DIF PREAMBLE
\RequirePackage{settobox} %DIF PREAMBLE
\RequirePackage{letltxmacro} %DIF PREAMBLE
\newsavebox{\DIFdelgraphicsbox} %DIF PREAMBLE
\newlength{\DIFdelgraphicswidth} %DIF PREAMBLE
\newlength{\DIFdelgraphicsheight} %DIF PREAMBLE
% store original definition of \includegraphics %DIF PREAMBLE
\LetLtxMacro{\DIFOincludegraphics}{\includegraphics} %DIF PREAMBLE
\newcommand{\DIFaddincludegraphics}[2][]{{\color{blue}\fbox{\DIFOincludegraphics[#1]{#2}}}} %DIF PREAMBLE
\newcommand{\DIFdelincludegraphics}[2][]{% %DIF PREAMBLE
\sbox{\DIFdelgraphicsbox}{\DIFOincludegraphics[#1]{#2}}% %DIF PREAMBLE
\settoboxwidth{\DIFdelgraphicswidth}{\DIFdelgraphicsbox} %DIF PREAMBLE
\settoboxtotalheight{\DIFdelgraphicsheight}{\DIFdelgraphicsbox} %DIF PREAMBLE
\scalebox{\DIFscaledelfig}{% %DIF PREAMBLE
\parbox[b]{\DIFdelgraphicswidth}{\usebox{\DIFdelgraphicsbox}\\[-\baselineskip] \rule{\DIFdelgraphicswidth}{0em}}\llap{\resizebox{\DIFdelgraphicswidth}{\DIFdelgraphicsheight}{% %DIF PREAMBLE
\setlength{\unitlength}{\DIFdelgraphicswidth}% %DIF PREAMBLE
\begin{picture}(1,1)% %DIF PREAMBLE
\thicklines\linethickness{2pt} %DIF PREAMBLE
{\color[rgb]{1,0,0}\put(0,0){\framebox(1,1){}}}% %DIF PREAMBLE
{\color[rgb]{1,0,0}\put(0,0){\line( 1,1){1}}}% %DIF PREAMBLE
{\color[rgb]{1,0,0}\put(0,1){\line(1,-1){1}}}% %DIF PREAMBLE
\end{picture}% %DIF PREAMBLE
}\hspace*{3pt}}} %DIF PREAMBLE
} %DIF PREAMBLE
\LetLtxMacro{\DIFOaddbegin}{\DIFaddbegin} %DIF PREAMBLE
\LetLtxMacro{\DIFOaddend}{\DIFaddend} %DIF PREAMBLE
\LetLtxMacro{\DIFOdelbegin}{\DIFdelbegin} %DIF PREAMBLE
\LetLtxMacro{\DIFOdelend}{\DIFdelend} %DIF PREAMBLE
\DeclareRobustCommand{\DIFaddbegin}{\DIFOaddbegin \let\includegraphics\DIFaddincludegraphics} %DIF PREAMBLE
\DeclareRobustCommand{\DIFaddend}{\DIFOaddend \let\includegraphics\DIFOincludegraphics} %DIF PREAMBLE
\DeclareRobustCommand{\DIFdelbegin}{\DIFOdelbegin \let\includegraphics\DIFdelincludegraphics} %DIF PREAMBLE
\DeclareRobustCommand{\DIFdelend}{\DIFOaddend \let\includegraphics\DIFOincludegraphics} %DIF PREAMBLE
\LetLtxMacro{\DIFOaddbeginFL}{\DIFaddbeginFL} %DIF PREAMBLE
\LetLtxMacro{\DIFOaddendFL}{\DIFaddendFL} %DIF PREAMBLE
\LetLtxMacro{\DIFOdelbeginFL}{\DIFdelbeginFL} %DIF PREAMBLE
\LetLtxMacro{\DIFOdelendFL}{\DIFdelendFL} %DIF PREAMBLE
\DeclareRobustCommand{\DIFaddbeginFL}{\DIFOaddbeginFL \let\includegraphics\DIFaddincludegraphics} %DIF PREAMBLE
\DeclareRobustCommand{\DIFaddendFL}{\DIFOaddendFL \let\includegraphics\DIFOincludegraphics} %DIF PREAMBLE
\DeclareRobustCommand{\DIFdelbeginFL}{\DIFOdelbeginFL \let\includegraphics\DIFdelincludegraphics} %DIF PREAMBLE
\DeclareRobustCommand{\DIFdelendFL}{\DIFOaddendFL \let\includegraphics\DIFOincludegraphics} %DIF PREAMBLE
%DIF COLORLISTINGS PREAMBLE %DIF PREAMBLE
\RequirePackage{listings} %DIF PREAMBLE
\RequirePackage{color} %DIF PREAMBLE
\lstdefinelanguage{DIFcode}{ %DIF PREAMBLE
%DIF DIFCODE_CFONT %DIF PREAMBLE
  moredelim=[il][\color{red}\scriptsize]{\%DIF\ <\ }, %DIF PREAMBLE
  moredelim=[il][\color{blue}\sffamily]{\%DIF\ >\ } %DIF PREAMBLE
} %DIF PREAMBLE
\lstdefinestyle{DIFverbatimstyle}{ %DIF PREAMBLE
	language=DIFcode, %DIF PREAMBLE
	basicstyle=\ttfamily, %DIF PREAMBLE
	columns=fullflexible, %DIF PREAMBLE
	keepspaces=true %DIF PREAMBLE
} %DIF PREAMBLE
\lstnewenvironment{DIFverbatim}{\lstset{style=DIFverbatimstyle}}{} %DIF PREAMBLE
\lstnewenvironment{DIFverbatim*}{\lstset{style=DIFverbatimstyle,showspaces=true}}{} %DIF PREAMBLE
\lstset{extendedchars=\true,inputencoding=utf8}

%DIF END PREAMBLE EXTENSION ADDED BY LATEXDIFF

\begin{document}

\title{大統一模型模型研究ノート}
\author{金沢大学大学院\,\,自然科学研究科数物科学専攻\,(物理学コース)修士課程2年\\学籍番号\,2315011026$\quad$名列番号216\\高村 泰時} 
\date{\today}
\maketitle

\begin{abstract}
  これは$SU(5)$大統一理論の研究のノートです.
\end{abstract}

\chapter{素粒子標準模型}
% ---------------------------------------
%
%  Standard model
%  Standard_Model.tex
%  Program modified by Yasutoki Takamura
%  Last Modified Jul 18 2024
%
% ---------------------------------------
この章では素粒子物理学における標準模型についてまとめた.
\section{標準模型}



%EOF


\chapter{素粒子標準模型の問題点}
% ---------------------------------------
%
%  Some difficulties of Standard Model
%  SM_problems.tex
%  Program modified by Yasutoki Takamura
%  Last Modified Jul 20 2024
%
% ---------------------------------------
\section{標準模型の抱える問題}
素粒子標準模型は高エネルギー物理学の実験をほぼ正確に予言することができるため, 大きな成功を収めた.
特に2011年にCERNにある大型ハドロン衝突型加速器(Large Hadron Corrider; LHC)が標準模型に現れるHiggs粒子を発見したことにより, 標準模型は揺るぎないものとなった.
しかし, 次のような課題があり, 理論の拡張が迫られている.
\begin{itemize}
        \item ニュートリノ質量, およびニュートリノ振動\\
              標準模型ではニュートリノは質量を持たない粒子として存在する.
              しかし1998年にニュートリノ振動がスーパーカミオカンデで観測されたことにより, ニュートリノは質量を持つことが示唆されたため, 標準模型を何かしら拡張する必要があると考えられている.
      \item 重力相互作用
      \item 真の統一理論
      \item 電荷の量子化\\
            素粒子の電荷は単位電荷の整数倍の値を持つ.
            非可換群の固有値であれば量子化が実現できるが, ハイパーチャージ$Y$は可換群である$U(1)$対称性における無限小演算子であり, 固有値$Y$量子化は行えない.
            したがって標準模型で電荷$Q$は$Q = I^3 + \frac{Y}{2}$という関係に基づいて決定されるが, この電荷が量子化される根拠は標準模型に存在しない.
      \item 階層性問題
      \item 予言できないパラメーターの数
      \item 暗黒物質
      \item 暗黒エネルギー
\end{itemize}
%EOF



\chapter{大統一理論}
\section{素粒子標準模型と大統一理論}
大統一理論はH.GeorgiとS.L.Glashowにより1974年に提唱された\cite{PhysRevLett.32.438}.


\chapter{階層性問題}
% ---------------------------------------
%
%  Hierarchy Problems
%  hierarchy_problem.tex
%  Program modified by Yasutoki Takamura
%  Last Modified Jul 20 2024
%
% ---------------------------------------
大統一理論や超弦理論を考えた場合, 一般的にエネルギースケールの階層性が問題となる.
ここでは大統一理論に表れる階層性に集中してこの問題について取り扱う.
\section{階層性問題とは}
% 1st version: Based on 素粒子の標準模型を超えて
素粒子標準模型は電弱スケールである$M_W\sim1000\,[\mathrm{GeV}]$まで高エネルギー加速器実験結果を説明することができる.
一方で標準模型を超えた物理(Beyond the Standard Model; BSM)が加速器実験で検証されるには電弱スケールよりも高いエネルギーにより, その実験を検証することが可能となる.

この見方を変えると, 現在の標準模型はこのようなBSMの有効理論であると考えることができる.
したがって素粒子標準模型の理論の適用範囲は何らかのエネルギースケールである$\Lambda$まで有効であり, $\Lambda$以上のエネルギーでは別の理論へ移り変わると考えられている.

大統一理論や重力が含まれる理論では, このカットオフは$\Lambda\sim M_{\mathrm{GUT}}$や$\Lambda\sim M_{\mathrm{pl}}$程度であるとそれぞれ考えられており, $M_W$に比べて13桁程度の乖離が存在する.

標準模型に登場する粒子はヒッグス粒子の真空期待値に比例するため, これらは電弱スケールに質量が存在することとなる.
これらはゲージ理論により説明されるが, ヒッグス粒子の質量を説明できる主導原理は標準模型に存在しない.
標準模型に表れるヒッグス粒子の質量を$m_h$とした場合, いかにして$m_h \ll \Lambda$を保つかが大きな問題となっている.

\section{Doublet-triplet splitting problem}
ここでは$SU(5)$大統一理論を考える.
$SU(5)$大統一理論では, 5表現ヒッグスと24表現ヒッグスを考えることができた.
それぞれ$H$,\,$\Phi$とおく.
これらのヒッグス粒子によるポテンシャルを考える.
$\mathbb{Z}_2$対称性を課すと,
\begin{align}
  V(H,\,\Phi) = -\frac{1}{2}\nu^2 H^\dagger H + \frac{\lambda}{4}(H^\dagger H)^2 + H^\dagger[\alpha \mathrm{Tr}(\Phi^2)+\beta(\Phi^2)]H \label{eq5-1} 
\end{align}
となる.
ここで, $\Phi$の最小化は式(\ref{eq5-1})の第3項の内部のみで行われていると考える.
これは式(\ref{eq5-1})は階層性のもとでは, 多項式全体の最小化の影響よりも, 十分影響を与えるためである.

ただし, このように真空期待値を取った場合, $Y$ボゾンに質量を与えうる$H^\alpha$と$\Phi^\alpha_5$という2つのカラー三重項ヒッグス場が存在したとしても片方のヒッグス場のみ質量を与え, もう一方は質量がないままとなる.

%EOF


\chapter{群論}
% ---------------------------------------
%
%  Group theory
%  group_theory.tex
%  Program modified by Yasutoki Takamura
%  Last Modified Jul 18 2024
%
% ----------------------------------------
 この章では, 大統一理論に必要な数学の内容をまとめている.
 詳しい証明や例については数学の専門書を参考にすること.
 次のことを認め, 話を進める.

 $X$を集合とする.
 写像$\phi: X\times X \rightarrow X$のことを集合$X$上の演算と言う.
 これ以降では$a,b,\in X$に対する写像を$\phi(a, b)$の代わりに$ab$と書く.
\section{群}
群とは次の性質を持つものである.
\begin{dfn}[群]
  $G$を空ではない集合とする. 集合$G$上で演算が定義されており, 次の性質を満たすとき, $G$を群と言う.
  \begin{enumerate}
    \item 単位元と呼ばれる$e\in G$が存在し, 全ての$a\in G$に対して$ae=ea=a$となる.
    \item すべての$a\in G$に対し, $b\in G$が存在し, $ab=ba=e$となる. この元$b$は$a$の逆元と呼ばれ, $a^{-1}$と書く.
    \item すべての$a, b, c \in G$に対して, $(ab)c=a(bc)$が成り立つ.
  \end{enumerate}
\end{dfn}
特に, 性質3. は結合法則と呼ばれている.
群の元$a, b\in G$に対して$ab=ba$が成り立つとき, $a, b$は可換である.
$G$の任意の元$a, b$が可換なら, $G$を可換群(Abel群)と呼ぶ.
\section{Lie群}


%EOF




% ----- Bibliography -----
\bibliographystyle{abbrv}
\begin{thebibliography}{1}

\bibitem{PhysRevLett.32.438}
H.~Georgi and S.~L. Glashow.
\newblock Unity of all elementary-particle forces.
\newblock {\em Phys. Rev. Lett.}, 32:438--441, Feb 1974.

\end{thebibliography}

% -----------------------
\end{document}

