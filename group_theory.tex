% ---------------------------------------
%
%  Group theory
%  group_theory.tex
%  Program modified by Yasutoki Takamura
%  Last Modified Jan 27 2025
%
% ----------------------------------------
この章では, 大統一理論に必要な数学の内容を非常に簡単にまとめた.
本文でもほとんど触れられているが, 数学の部分のみを取り出してまとめている.
次のことを認め, 話を進める.

$X$を集合とする.
写像$\phi: X\times X \rightarrow X$のことを集合$X$上の演算と言う.
これ以降では$a,b,\in X$に対する写像を$\phi(a, b)$の代わりに$ab$と書く.
\section{群}
群とは次の性質を持つものである.
\begin{dfn}[群]
  $G$を空ではない集合とする. 集合$G$上で演算が定義されており, 次の性質を満たすとき, $G$を群と言う.
  \begin{enumerate}
    \item 単位元と呼ばれる$e\in G$が存在し, 全ての$a\in G$に対して$ae=ea=a$となる.
    \item すべての$a\in G$に対し, $b\in G$が存在し, $ab=ba=e$となる. この元$b$は$a$の逆元と呼ばれ, $a^{-1}$と書く.
    \item すべての$a, b, c \in G$に対して, $(ab)c=a(bc)$が成り立つ.
  \end{enumerate}
\end{dfn}
特に, 性質3. は結合法則と呼ばれている.
群の元$a, b\in G$に対して$ab=ba$が成り立つとき, $a, b$は可換である.
$G$の任意の元$a, b$が可換なら, $G$を可換群(Abel群)と呼ぶ.
\section{Lie群}
Lie群は連続なパラメータにより特徴づけられる群であり, 生成子$t^a$によって
\begin{align}
  g(\alpha) = \exp(i\alpha^a t^a)
\end{align}
となる.
この生成子$t^a$は, リー括弧によって
\begin{align}
  [t^a, t^b] = if^{abc}t^c
\end{align}
を満たす.
ここで, $f^{abc}$はリー代数の構造定数である.
\section{表現}
表現とは, 群を行列に対応させる写像である.
この写像は$\mathcal{R}: G\rightarrow GL(n)$で定義される.
表現の次元は行列の次元と等しい.


%EOF

