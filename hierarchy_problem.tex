% ---------------------------------------
%
%  Hierarchy Problems
%  hierarchy_problem.tex
%  Program modified by Yasutoki Takamura
%  Last Modified Jul 20 2024
%
% ---------------------------------------
大統一理論や超弦理論を考えた場合, 一般的にエネルギースケールの階層性が問題となる.
ここでは大統一理論に表れる階層性に集中してこの問題について取り扱う.
\subsection{階層性問題とは}
% 1st version: Based on 素粒子の標準模型を超えて
素粒子標準模型は電弱スケールである$M_W\sim100\,[\mathrm{GeV}]$まで高エネルギー加速器実験結果を説明することができる.
一方で標準模型を超えた物理(Beyond the Standard Model; BSM)が加速器実験で検証されるには電弱スケールよりも高いエネルギーにより, その実験を検証することが可能となる.

この見方を変えると, 現在の標準模型はこのようなBSMの有効理論であると考えることができる.
したがって素粒子標準模型の理論の適用範囲は何らかのエネルギースケールである$\Lambda$まで有効であり, $\Lambda$以上のエネルギーでは別の理論へ移り変わると考えられている.

大統一理論や重力が含まれる理論では, このカットオフは$\Lambda\sim M_{\mathrm{GUT}}$や$\Lambda\sim M_{\mathrm{pl}}$程度であるとそれぞれ考えられており, $M_W$に比べて13桁程度の乖離が存在する.

標準模型に登場する粒子はヒッグス粒子の真空期待値に比例するため, これらは電弱スケールに質量が存在することとなる.
これらはゲージ理論により説明されるが, ヒッグス粒子の質量を説明できる主導原理は標準模型に存在しない.
標準模型に表れるヒッグス粒子の質量を$m_h$とした場合, いかにして$m_h \ll \Lambda$を保つかが大きな問題となっている.

\subsection{Doublet-triplet splitting problem}
ここでは$SU(5)$大統一理論を考える.
$SU(5)$大統一理論では, 5表現ヒッグスと24表現ヒッグスを考えることができた.
それぞれ$H$,\,$\Phi$とおく.
これらのヒッグス粒子によるポテンシャルを考える.
$\mathbb{Z}_2$対称性を課すと,
\begin{align}
  V(H,\,\Phi) = -\frac{1}{2}\nu^2 H^\dagger H + \frac{\lambda}{4}(H^\dagger H)^2 + H^\dagger[\alpha \mathrm{Tr}(\Phi^2)+\beta(\Phi^2)]H \label{eq5-1} 
\end{align}
となる.
ここで, $\Phi$の最小化は式(\ref{eq5-1})の第3項の内部のみで行われていると考える.
これは式(\ref{eq5-1})は階層性のもとでは, 多項式全体の最小化の影響よりも, 十分影響を与えるためである.

ただし, このように真空期待値を取った場合, $Y$ボゾンに質量を与えうる$H^\alpha$と$\Phi^\alpha_5$という2つのカラー三重項ヒッグス場が存在したとしても片方のヒッグス場のみ質量を与え, もう一方は質量がないままとなる.

%EOF
