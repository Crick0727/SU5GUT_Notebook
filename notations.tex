% ---------------------------------------
%
%  Notations
%  notations.tex
%  Program modified by Yasutoki Takamura
%  Last Modified Jan 16 2025
%
% ---------------------------------------
この本で用いる記法については次のものにしたがう.
\section{場の量子論に関係する}
計量テンソル
\begin{align}
  g_{\mu\nu} = g^{\mu\nu} = \mathrm{diag}(1, -1, -1, -1)
\end{align}

パウリ行列
\begin{align}
  \sigma ^1 = \left(\begin{array}{cc}
        0 & 1 \\
        1 & 0 
      \end{array}\right),\quad
  \sigma ^2 = \left(\begin{array}{cc}
        0 & -i \\
        i &  0 
      \end{array}\right),\quad
  \sigma ^3 = \left(\begin{array}{cc}
        1 & 0 \\
        0 & -1 
      \end{array}\right)
\end{align}

\section{SU(5)大統一理論}

$\bar{\bf{5}}$表現フェルミオン
\begin{align}
  \psi_{\bar{\bm{5}}} =\begin{pmatrix}
    d_1 ^c \\
    d_2 ^c \\
    d_3 ^c \\
    l      \\
    -\nu_l
  \end{pmatrix}_L
  =\left(\bar{3},1,\frac{1}{3}\right)\oplus \left(1,2,-\frac{1}{2}\right)
\end{align}

$\bm{10}$表現フェルミオン
\begin{align}
  \psi_{{\bm{10}}} = \frac{1}{\sqrt{2}}\begin{pmatrix}
         0 &  u_3^c & -u_2^c & u_1 & d_1 \\
    -u_3^c &      0 &  u_1^c & u_2 & d_2 \\
     u_2^c & -u_1^c &      0 & u_3 & d_3 \\
    -u_1   &   -u_2 &   -u_3 &   0 & e^c \\
      -d_1 &   -d_2 &   -d_3 &-e^c &   0 \\
    \end{pmatrix}_L = (3,2,,\frac{1}{6})\oplus(\bar{3},1,-\frac{2}{3})\oplus(1,1,1)
\end{align}

%EOF
