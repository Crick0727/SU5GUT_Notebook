% ---------------------------------------
%
%  Hierarchy Problems
%  hierarchy_problem.tex
%  Program modified by Yasutoki Takamura
%  Last Modified Jul 18 2024
%
% ---------------------------------------
\section{階層性問題}
素粒子標準模型は電弱スケールである$M_W\sim1000\,[\mathrm{GeV}]$まで高エネルギー加速器実験結果を説明することができる.
一方で標準模型を超えた物理(Beyond the Standard Model;BSM)が加速器実験で検証されるには電弱スケールよりも高いエネルギーにより, その実験を検証することが可能となる.

この見方を変えると, 現在の標準模型はこのようなBSMの有効理論であると考えることができる.
したがって素粒子標準模型の理論の適用範囲は何らかのエネルギースケールである$\Lambda$まで有効であり, $\Lambda$以上のエネルギーでは別の理論へ移り変わると考えられている.

