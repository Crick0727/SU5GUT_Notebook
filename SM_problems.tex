% ---------------------------------------
%
%  Some difficulties of Standard Model
%  SM_problems.tex
%  Program modified by Yasutoki Takamura
%  Last Modified Nov 26 2024
%
% ---------------------------------------
\section{標準模型の抱える問題}
素粒子標準模型は高エネルギー物理学の実験をほぼ正確に予言することができるため, 大きな成功を収めた.
特に2011年にCERNにある大型ハドロン衝突型加速器(Large Hadron Corrider; LHC)が標準模型に現れるHiggs粒子を発見したことにより, 標準模型は揺るぎないものとなった.
しかし, 次のような課題があり, 理論の拡張が迫られている.
\begin{itemize}
        \item ニュートリノ質量, およびニュートリノ振動\\
              標準模型ではニュートリノは質量を持たない粒子として存在する.
              しかし1998年にニュートリノ振動がスーパーカミオカンデで観測されたことにより, ニュートリノは質量を持つことが示唆されたため, 標準模型を何かしら拡張する必要があると考えられている.
      \item 重力相互作用\\
            粒子の持つ相互作用のうち, 重力相互作用は他の相互作用と異なり, 一般相対論で記述されるものであり, これを量子化しようとするとくりこみができない問題が生じる.
      \item 真の統一理論\\
            標準模型はゲージ対称性を$SU(3)_c\times SU(2)_L\times U(1)_Y$とした理論であるが, 直積としてゲージ群が記述されていることは群の操作は独立に行われていることを意味しており, 真の統一理論とは言えない.
      \item 電荷の量子化\\
            素粒子の電荷は単位電荷の整数倍の値を持つ.
            非可換群の固有値であれば量子化が実現できるが, ハイパーチャージ$Y$は可換群である$U(1)$対称性における無限小演算子であり, 固有値$Y$量子化は行えない.
            したがって標準模型で電荷$Q$は$Q = I^3 + \frac{Y}{2}$という関係に基づいて決定されるが, この電荷が量子化される根拠は標準模型に存在しない.
      \item 階層性問題\\
            標準模型の典型的なエネルギースケールは$M_W\simeq 80\text{GeV}$である.
	    プランクスケール$M_{\mathrm{pl}}$や大統一スケール$M_{\mathrm{GUT}}$に至れば理論が標準模型と入れ替わると考えられているが, エネルギースケールに大きな隔たりがあり, この階層性を如何にして自然と説明できるかは重要な問題である.
            階層性問題は大統一理論を実現するスケールや, 自発的に対称性を破るヒッグス粒子とも深く関わる.
            そのため次章で大統一理論による問題点に注目して改めて述べる.
      \item 予言できないパラメーターの数\\
            素粒子標準模型はゲージ相互作用と量子場の理論で構成されるが, 唯一のスカラー場であるヒッグス粒子は標準模型であっても相互作用を規定する主導原理は存在せず, クォークやレプトンとの湯川相互作用やヒッグス自身の自己相互作用は理論の中では任意の値をとることが可能であり, 予言することができない.
      \item 暗黒物質\\
            標準模型に存在する粒子のみを考えた場合, 観測事実として宇宙全体のエネルギーに対して$4\%$のみしか説明することができず, 残りの$96\%$のうち$23\%$は暗黒物質と考えられている.
	    ここで言う暗黒物質とは, 物質粒子との相互作用が極めて弱いものの重力相互作用を微弱に持つものであると考えられているが, そのような粒子は素粒子標準模型のみでは説明することができない.
      \item 暗黒エネルギー\\
            先に述べた暗黒物質を考えた場合であっても, およそ$73\%$は未知のエネルギーとして考えられており, これは暗黒エネルギーと呼ばれる.
	    暗黒エネルギーは初期宇宙においてインフレーションと呼ばれる加速膨張を引き起こした宇宙項と関係があると考えられている
\end{itemize}
%EOF

