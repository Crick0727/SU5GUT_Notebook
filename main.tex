\documentclass[titlepage]{jsbook}

% ----- preambles ------------------
\usepackage{geometry}
\usepackage{bm}
\usepackage{here}
\usepackage{amsmath,amsthm,amssymb}
\usepackage{ascmac}
\usepackage[dvipdfmx]{graphicx}
% ----------------------------------

% ----- Settings for amsthm -----
\theoremstyle{plain}
\newtheorem{thm}{Theorem}
\newtheorem*{thm*}{Theorem}

\theoremstyle{definition}
\newtheorem{dfn}{Definition}
% -------------------------------

% ----- Geometry setting ---------------------------
\geometry{left=25mm,right=25mm,top=30mm,bottom=30mm}
% ---------------------------------------------------
\begin{document}

\title{大統一模型模型研究ノート}
\author{金沢大学大学院\,\,自然科学研究科数物科学専攻\,(物理学コース)修士課程2年\\学籍番号\,2315011026$\quad$名列番号216\\高村 泰時} 
\date{\today}
\maketitle

\begin{abstract}
  これは$SU(5)$大統一理論の研究のノートです.
\end{abstract}

\chapter{素粒子標準模型}
% ---------------------------------------
%
%  Standard model
%  Standard_Model.tex
%  Program modified by Yasutoki Takamura
%  Last Modified Nov 29 2024
%
% ---------------------------------------
この章では素粒子物理学における標準模型についてまとめた.
素粒子標準模型は現在の高エネルギー実験をほぼ説明することができるものであるが, 標準模型だけでは説明できない事実が存在する.
大統一理論はそれらを解決するための理論の一つであるが, そもそも標準模型がどのような理論であるのかをここで振り返る.
\subsection{概観}
素粒子標準模型は量子場の理論により記述される.
素粒子の相互作用はYang-Millsによる理論によって, 数学的にリー群に属する局所非可換ゲージ変換のもとで不変になるようにラグランジアン密度を記述することができる.
具体的には, 次のようなゲージ対称性を持つ理論である.
\begin{align}
  \mathcal{G}_{\text{SM}}\mathrm{SU}_\mathrm{c}\times \mathrm{SU}_\mathrm{L}\times \mathrm{U}(1)_\mathrm{Y}\label{SM-Gauge}
\end{align}
それぞれゲージ群の添字は量子数に対応している.
"C"はカラー量子数, "L"は左巻きのカイラリティ, "Y"は超電荷である.
$\mathrm{SU}3_c$群は核子にはたらく強い相互作用を記述する群であり, 量子色力学(QCD)によって記述される.
一方で$\mathrm{SU}(2)_L\times \mathrm{U}(1)_Y$は弱い相互作用と電磁相互作用を統一的に記述する電弱理論を記述する群である.

ゲージ対称性の他に, 理論に現れる粒子がどのようなものであるのか, 標準模型のゲージ対称性の中でどのような対称性を持つ粒子であるのかを決めることにより, 理論を具体的に決定することができる.
他にも, ラグランジアン密度に含める項は, 繰り込み可能であり, ローレンツ対称性を満足するものを決めることで全て書き下すことができる.
以下では, 標準模型に現れる粒子について具体的に記述をする.
\subsection{ゲージ粒子}
\subsection{フェルミオン}
\subsection{ゲージ理論}
\subsection{ヒッグス機構}


%EOF


\chapter{素粒子標準模型の問題点}
% ---------------------------------------
%
%  Some difficulties of Standard Model
%  SM_problems.tex
%  Program modified by Yasutoki Takamura
%  Last Modified Jul 18 2024
%
% ---------------------------------------
\section{標準模型の抱える問題}
素粒子標準模型は高エネルギー物理学の実験をほぼ正確に予言することができるため, 大きな成功を収めた.
特に2011年にCERNにある大型ハドロン衝突型加速器(Large Hadron Corrider; LHC)が標準模型に現れるHiggs粒子を発見したことにより, 標準模型は揺るぎないものとなった.
しかし, 次のような課題があり, 理論の拡張が迫られている.
\begin{itemize}
        \item ニュートリノ質量, およびニュートリノ振動\\
              標準模型ではニュートリノは質量を持たない粒子として存在する.
              しかし1998年にニュートリノ振動がスーパーカミオカンデで観測されたことにより, ニュートリノは質量を持つことが示唆されたため, 標準模型を何かしら拡張する必要があると考えられている.
      \item 重力相互作用
      \item 真の統一理論
      \item 電荷の量子化
      \item 階層性問題
      \item 予言できないパラメーターの数
      \item 暗黒物質
      \item 暗黒エネルギー
\end{itemize}
%EOF



\chapter{大統一理論}
% ----------------------------------------
% 
%  Grand Unified Theory
%  grand_unified_theory.tex
%  Program modified by Yasutoki Takamura
%  Last Modified Jan 10 2025
%
% ----------------------------------------
%\section{大統一理論}
大統一理論はH.GeorgiとS.L.Glashowにより1974年に提唱された\cite{PhysRevLett.32.438}.
大統一理論では重力を除いた3つの相互作用を1つに統一することを目的としている.
したがって, ゲージ対称性は単純群によって記述されると考えられており, 標準模型のゲージ群である$\mathcal{G}_\text{SM}= SU(3)_c\times SU(2)_L\times U(1)_Y$を部分群として内包する群を考える.
このことから, 前節で述べられているような標準模型の問題点はいくつか解決されると考えられている.

\textcolor{red}{要修正: どの点で何が解決できているのか}

大統一理論にも様々な単純群を考えることが可能であり, $SU(5)$や$SO(10)$, $E_6$などの模型を考えることが可能であるが, このノートでは最小模型である$SU(5)$について取り扱い, 理論の拡張を試みる.

はじめに, $SU(5)$大統一理論が最小模型である理由を考える.
標準模型のゲージ対称性である$SU(3)_c\times SU(2)_L\times U(1)_Y$は, 群の階数が$\text{rank}=4$であり, これを内包できる群を考える必要がある.
その中で考えられる単純群は, $\text{O}(8)$, $\mathrm{O}(9)$, $\mathrm{Sp}(8)$, $\mathrm{F}_4$, そして $\mathrm{SU}(5)$である.
このうち, $SU(3)$のもつ3重項をもち, $SU(2)$がもつ2重項の複素表現をもつという性質は$SU(5)$群を用いて記述することができるため, 最小模型として$SU(5)$群を考えて理論を構築することが可能となる.

\section{Georgi-Glashow モデル}
ここから$SU(5)$群を考える.
ゲージ群を決定すると, ラグランジアンに導入する場の既約表現を指定すれば理論を定めることができる.
$SU(5)$の基本表現は$5$表現であり, $5$表現か, その複素表現である$\overline{5}$を用いることですべての表現を構成することができる.
$SU(5)$群の階数は4であるから, 生成演算子を$L_i$$(i=1,\cdots,24)$ のうち対角化可能なものが4つ存在する.
$SU(3)_c$の対角化可能な生成演算子を$\lambda_3,\lambda_8$, $\mathrm{SU}(2)$のものを$I_3$ (アイソスピン), $U(1)$の超電荷を$Y$に対応させて考えることができる.

次にフェルミオンの次元を$\mathrm{SU}(3), \mathrm{SU}(2)$の表現次元, $U(1)$の超電荷$Y$で $(1, 2, -1)$のように表し, どのように$SU(5)$模型に当てはめられるか考える.
\begin{center}
\begin{tabular}{clc}
             &                         & 多重度 \\
  $(u, d)_L$ & $(3  , 2, \frac{1}{6})$ &      6 \\
    $d_L^c $ & $(3^*, 1, \frac{1}{3})$ &      3 \\ 
    $u_L^c $ & $(3^*, 1,-\frac{2}{3})$ &      3 \\ 
$(\nu, e)_L$ & $(1  , 2,-\frac{1}{2})$ &      2 \\ 
     $e_L^c$ & $(1  , 1,           1)$ &      1 \\ 
   $\nu_L^c$ & $(1  , 1,           0)$ &      1 \\ 
\end{tabular}
\end{center}

基本表現をこの中から選ぶことで
\begin{align}
  (q_L^c, \nu_L, e_L) = (3^*, 1, Y)\oplus\left(1,2,-\frac{1}{2}\right) = \overline{5}
\end{align}
と考えられる.
ここではまだ$q$の正体が$u, d$クォークのどちらかは定かではないが, $\mathrm{SU}(5)$群のトレースは0でなければならないため, $Q$を電荷とすると上記の電荷は
\begin{align}
  \sum_{a=1}^5 Q_a = 3Q_{qc}+Q_\nu + Q_e\nonumber\\
  \therefore\quad Q_q = -Q_{qc} = -\frac{1}{3}
\end{align}
となる.
したがって, $q=d$と選択することにより, 次のように$\bm{5}$表現を決めることができる.
\begin{align}
 \bar{\bm{5}}=\begin{pmatrix}
    d_1 ^c \\
    d_2 ^c \\
    d_3 ^c \\
    l      \\
    -\nu_l
  \end{pmatrix}=\left({3}^*,1,\frac{1}{3}\right)\oplus \left(1,2,-\frac{1}{2}\right)\label{GUT-5rep}
\end{align}
これによりクォークの電荷が荷電レプトンの$\frac{1}{3}$の整数倍となっていることが説明できる.
残りのフェルミオンを次元の大きい表現に当てはめることを考える.
基本表現よりも大きな表現は, すべて基本表現の積で表すことができる.
${\bm{10}}$表現は
\begin{align}
  \bm{5}\otimes\bm{5} = \bm{15}\oplus\bm{10}\nonumber
\end{align}
として構成することができる.
この$\bf{10}$表現は$\bm{10}\rightarrow (3^*,1,-\frac{2}{3})\oplus(3,2,\frac{1}{6})\oplus(1,1,1)$
と分解できるので, 次のように当てはめることができる.
\begin{align}
  \overline{\bm{10}}= \frac{1}{\sqrt{2}}\begin{pmatrix}
         0 &  u_3^c & -u_2^c & u_1 & d_1 \\
    -u_3^c &      0 &  u_1^c & u_2 & d_2 \\
     u_2^c & -u_1^c &      0 & u_3 & d_3 \\
    -u_1   &   -u_2 &   -u_3 &   0 & e^c \\
      -d_1 &   -d_2 &   -d_3 &-e^c &   0 \\
    \end{pmatrix}\label{GUT-10rep}
% Note
% 10表現であれば内部に規格化定数を入れる場合があるがここで必要ではないのか確認しなければならない.
% Dec 10: up quark massを出す際に規格化されている必要があるので, この規格化定数は必要.
\end{align}
ただし, $x^c$は$x$を同じカイラリティで荷電共役したものを表している.
この$x$の例として, 左巻きの電子$e^-_L$を考える.
$e^-_L$に対して荷電共役変換を行った$(e^-_L)^c$を考える.
これは右巻きの電子$e_R^-$となるが, 左巻きの陽電子である$e_L^+$と同じである.
\subsection{ゲージ粒子の表現}
$SU(n)$ゲージ理論では, $n^2-1$個のゲージ場が存在する.
$SU(5)$群の生成演算子を, $[L_i]^a_b\,(i=1,\cdots,24, a,b=1,\cdots,5)$として構成する.
基本表現である$5$表現の類推からは$a,b=1,2,3$はカラー量子数, $a,b=4,5$がアイソスピン量子数と考えられる.
そのため$i=1,\cdots,8$を$SU(3)$部分, $i=9,10,11$を$SU(2)$の部分群として考えると
\begin{align}
  L_i(i=1,\cdots,8) = \left(\begin{array}{ccccc}
        &           &   & 0 & 0 \\
        & \lambda_i &   & 0 & 0 \\
        &           &   & 0 & 0 \\
      0 &         0 & 0 & 0 & 0 \\
      0 &         0 & 0 & 0 & 0 
  \end{array}\right),\quad
  L_i(i=9,10,11) = \left(\begin{array}{ccclcc}
      0 &  0 & 0 & & 0 & 0 \\
      0 &  0 & 0 & & 0 & 0 \\
      0 &  0 & 0 & & 0 & 0 \\
      0 &  0 & 0 & & \sigma_i      \\
      0 &  0 & 0 & &       \\
  \end{array}\right),\quad
\end{align}
24個のゲージボゾンは随伴表現として存在し, 次のように分解できる.
\begin{align}
  \bm{24} = G^i(8,1,0)\oplus W^i(1,3,0)\oplus B(1,1,0)\oplus \bar{X}(3,\bar{2},-\frac{5}{6})\oplus X(\bar{3},2,+\frac{5}{6})
\end{align}
ゲージボゾンを$V_\mu$とすると,
\begin{align}
  \frac{1}{\sqrt{2}}V_\mu &= \sum_{i=1}^{24}\frac{1}{2}V_\mu^i L_i\\
  V_\mu &= \left(\begin{array}{ccccc}
    G^1_1 - \frac{2B}{\sqrt{30}} & G^1_2 & G^1_3 & \bar{X_1} & \bar{Y_1} \\
    G^2_1 & G^2_2 - \frac{2B}{\sqrt{30}} & G^2_3 & \bar{X_2} & \bar{Y_2} \\
    G^3_1 & G^3_2 & G^3_3 - \frac{2B}{\sqrt{30}} & \bar{X_3} & \bar{Y_3} \\
    X_1   & X_2 & X_3 & \frac{W^3}{\sqrt{2}} + \frac{3B}{\sqrt{30}} & W^+ \\
    Y_1   & Y_2 & Y_3 & W^- &    -\frac{W^3}{\sqrt{2}} + \frac{3B}{\sqrt{30}} 
      \end{array}
  \right)
\end{align}
ただし, $SU(3)$ゲージ場の部分は$G^i\,(i=1,\cdots,8)$をグルーオン場として
\begin{align}
  G^1_2 = G^1,\quad G^2_1 = G^2,\quad G^3_1 = G^5,\quad G^2_3 = G^6,\quad G^3_2 = G_7\nonumber\\
  G^1_1 = \frac{G^3}{\sqrt{2}} + \frac{G^8}{\sqrt{6}},\quad G^2_2=-\frac{G^3}{\sqrt{2}}+\frac{G^8}{\sqrt{6}},\quad G^3_3 = -\frac{2G^8}{\sqrt{6}}\nonumber
\end{align}
であり, $SU(2)$ゲージ場は
\begin{align}
  W^\pm = \frac{W^1\mp iW^2}{\sqrt{2}}\nonumber
\end{align}
である.
\subsection{対称性の破れ}
\subsection{予言される現象}
\subsection{破綻}


%EOF


\chapter{階層性問題}
% ---------------------------------------
%
%  Hierarchy Problems
%  hierarchy_problem.tex
%  Program modified by Yasutoki Takamura
%  Last Modified Jul 20 2024
%
% ---------------------------------------
大統一理論や超弦理論を考えた場合, 一般的にエネルギースケールの階層性が問題となる.
ここでは大統一理論に表れる階層性に集中してこの問題について取り扱う.
\subsection{階層性問題とは}
% 1st version: Based on 素粒子の標準模型を超えて
素粒子標準模型は電弱スケールである$M_W\sim100\,[\mathrm{GeV}]$まで高エネルギー加速器実験結果を説明することができる.
一方で標準模型を超えた物理(Beyond the Standard Model; BSM)が加速器実験で検証されるには電弱スケールよりも高いエネルギーにより, その実験を検証することが可能となる.

この見方を変えると, 現在の標準模型はこのようなBSMの有効理論であると考えることができる.
したがって素粒子標準模型の理論の適用範囲は何らかのエネルギースケールである$\Lambda$まで有効であり, $\Lambda$以上のエネルギーでは別の理論へ移り変わると考えられている.

大統一理論や重力が含まれる理論では, このカットオフは$\Lambda\sim M_{\mathrm{GUT}}$や$\Lambda\sim M_{\mathrm{pl}}$程度であるとそれぞれ考えられており, $M_W$に比べて13桁程度の乖離が存在する.

標準模型に登場する粒子はヒッグス粒子の真空期待値に比例するため, これらは電弱スケールに質量が存在することとなる.
これらはゲージ理論により説明されるが, ヒッグス粒子の質量を説明できる主導原理は標準模型に存在しない.
標準模型に表れるヒッグス粒子の質量を$m_h$とした場合, いかにして$m_h \ll \Lambda$を保つかが大きな問題となっている.

\subsection{Doublet-triplet splitting problem}
ここでは$SU(5)$大統一理論を考える.
$SU(5)$大統一理論では, 5表現ヒッグスと24表現ヒッグスを考えることができた.
それぞれ$H$,\,$\Phi$とおく.
これらのヒッグス粒子によるポテンシャルを考える.
$\mathbb{Z}_2$対称性を課すと,
\begin{align}
  V(H,\,\Phi) = -\frac{1}{2}\nu^2 H^\dagger H + \frac{\lambda}{4}(H^\dagger H)^2 + H^\dagger[\alpha \mathrm{Tr}(\Phi^2)+\beta(\Phi^2)]H \label{eq5-1} 
\end{align}
となる.
ここで, $\Phi$の最小化は式(\ref{eq5-1})の第3項の内部のみで行われていると考える.
これは式(\ref{eq5-1})は階層性のもとでは, 多項式全体の最小化の影響よりも, 十分影響を与えるためである.

ただし, このように真空期待値を取った場合, $Y$ボゾンに質量を与えうる$H^\alpha$と$\Phi^\alpha_5$という2つのカラー三重項ヒッグス場が存在したとしても片方のヒッグス場のみ質量を与え, もう一方は質量がないままとなる.

%EOF


\chapter{群論}
% ---------------------------------------
%
%  Group theory
%  group_theory.tex
%  Program modified by Yasutoki Takamura
%  Last Modified Jan 27 2025
%
% ----------------------------------------
この章では, 大統一理論に必要な数学の内容を非常に簡単にまとめた.
本文でもほとんど触れられているが, 数学の部分のみを取り出してまとめている.
次のことを認め, 話を進める.

$X$を集合とする.
写像$\phi: X\times X \rightarrow X$のことを集合$X$上の演算と言う.
これ以降では$a,b,\in X$に対する写像を$\phi(a, b)$の代わりに$ab$と書く.
\section{群}
群とは次の性質を持つものである.
\begin{dfn}[群]
  $G$を空ではない集合とする. 集合$G$上で演算が定義されており, 次の性質を満たすとき, $G$を群と言う.
  \begin{enumerate}
    \item 単位元と呼ばれる$e\in G$が存在し, 全ての$a\in G$に対して$ae=ea=a$となる.
    \item すべての$a\in G$に対し, $b\in G$が存在し, $ab=ba=e$となる. この元$b$は$a$の逆元と呼ばれ, $a^{-1}$と書く.
    \item すべての$a, b, c \in G$に対して, $(ab)c=a(bc)$が成り立つ.
  \end{enumerate}
\end{dfn}
特に, 性質3. は結合法則と呼ばれている.
群の元$a, b\in G$に対して$ab=ba$が成り立つとき, $a, b$は可換である.
$G$の任意の元$a, b$が可換なら, $G$を可換群(Abel群)と呼ぶ.
\section{Lie群}
Lie群は連続なパラメータにより特徴づけられる群であり, 生成子$t^a$によって
\begin{align}
  g(\alpha) = \exp(i\alpha^a t^a)
\end{align}
となる.
この生成子$t^a$は, リー括弧によって
\begin{align}
  [t^a, t^b] = if^{abc}t^c
\end{align}
を満たす.
ここで, $f^{abc}$はリー代数の構造定数である.
\section{表現}
表現とは, 群を行列に対応させる写像である.
この写像は$\mathcal{R}: G\rightarrow GL(n)$で定義される.
表現の次元は行列の次元と等しい.


%EOF




% ----- Bibliography -----
\bibliographystyle{abbrv}
\bibliography{bibs.bib}
% -----------------------
\end{document}

