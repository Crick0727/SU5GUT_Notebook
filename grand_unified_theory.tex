% ----------------------------------------
% 
%  Grand Unified Theory
%  grand_unified_theory.tex
%  Program modified by Yasutoki Takamura
%  Last Modified Nov 28 2024
%
% ----------------------------------------

\section{大統一理論}
大統一理論はH.GeorgiとS.L.Glashowにより1974年に提唱された\cite{PhysRevLett.32.438}.
大統一理論では重力を除いた3つの相互作用を1つに統一することを目的としている.
したがって, ゲージ対称性は単純群によって記述されると考えられており, 標準模型のゲージ群である$\mathcal{G}_\text{SM}= SU(3)_c\times SU(2)_L\times U(1)_Y$を部分群として内包する群を考える.
このことから, 前節で述べられているような標準模型の問題点はいくつか解決されると考えられている.

\textcolor{red}{要修正: どの点で何が解決できているのか}

大統一理論にも様々な単純群を考えることが可能であり, $SU(5)$や$SO(10)$, $E_6$などの模型を考えることが可能であるが, このノートでは最小模型である$SU(5)$について取り扱い, 理論の拡張を試みる.

はじめに, $SU(5)$大統一理論が最小模型である理由を考える.
標準模型のゲージ対称性である$SU(3)_c\times SU(2)_L\times U(1)_Y$は, 群の階数が$\text{rank}=4$であり, これを内包できる群を考える必要がある.
その中で考えられる単純群は, $\text{O}(8)$, $\mathrm{O}(9)$, $\mathrm{Sp}(8)$, $\mathrm{F}_4$, そして $\mathrm{SU}(5)$である.
このうち, $SU(3)$のもつ3重項をもち, $SU(2)$がもつ2重項の複素表現をもつという性質は$SU(5)$群を用いて記述することができるため, 最小模型として$SU(5)$群を考えて理論を構築することが可能となる.

\subsection{Georgi-Glashow モデル}
ここから$SU(5)$群を考える.
ゲージ群を決定すると, ラグランジアンに導入する場の既約表現を指定すれば理論を定めることができる.
$SU(5)$の基本表現は$5$表現であり, $5$表現か, その複素表現である$\overline{5}$を用いることですべての表現を構成することができる.
$SU(5)$群の階数は4であるから, 生成演算子を$L_i$$(i=1,\cdots,24)$ のうち対角化可能なものが4つ存在する.
$SU(3)_c$の対角化可能な生成演算子を$\lambda_3,\lambda_8$, $\mathrm{SU}(2)$のものを$I_3$ (アイソスピン), $U(1)$の超電荷を$Y$に対応させて考えることができる.

次にフェルミオンの次元を$\mathrm{SU}(3), \mathrm{SU}(2)$の表現次元, $U(1)$の超電荷$Y$で $(1, 2, -1)$のように表し, どのように$SU(5)$模型に当てはめられるか考える.
\begin{center}
\begin{tabular}{clc}
             &                         & 多重度 \\
  $(u, d)_L$ & $(3  , 2, \frac{1}{6})$ &      6 \\
    $d_L^c $ & $(3^*, 1, \frac{1}{3})$ &      3 \\ 
    $u_L^c $ & $(3^*, 1,-\frac{2}{3})$ &      3 \\ 
$(\nu, e)_L$ & $(1  , 2,-\frac{1}{2})$ &      2 \\ 
     $e_L^c$ & $(1  , 1,           1)$ &      1 \\ 
   $\nu_L^c$ & $(1  , 1,           0)$ &      1 \\ 
\end{tabular}
\end{center}

基本表現をこの中から選ぶことで
\begin{align}
  (q_L^c, \nu_L, e_L) = (3^*, 1, Y)\oplus\left(1,2,-\frac{1}{2}\right) = \overline{5}
\end{align}
と考えられる.
ここではまだ$q$の正体が$u, d$クォークのどちらかは定かではないが, $\mathrm{SU}(5)$群のトレースは0でなければならないため, $Q$を電荷とすると上記の電荷は
\begin{align}
  \sum_{a=1}^5 Q_a = 3Q_{qc}+Q_\nu + Q_e\nonumber\\
  \therefore\quad Q_q = -Q_{qc} = -\frac{1}{3}
\end{align}
となる.
したがって, $q=d$と選択することにより, 次のように$\bm{5}$表現を決めることができる.
\begin{align}
 \bar{\bm{5}}=\begin{pmatrix}
    d_1 ^c \\
    d_2 ^c \\
    d_3 ^c \\
    l      \\
    -\nu_l
  \end{pmatrix}=\left({3}^*,1,\frac{1}{3}\right)\oplus \left(1,2,-\frac{1}{2}\right)\label{GUT-5rep}
\end{align}
これによりクォークの電荷が荷電レプトンの$\frac{1}{3}$の整数倍となっていることが説明できる.
残りのフェルミオンを次元の大きい表現に当てはめることを考える.
基本表現よりも大きな表現は, すべて基本表現の積で表すことができる.
${\bm{10}}$表現は
\begin{align}
  \bm{5}\otimes\bm{5} = \bm{15}\oplus\bm{10}\nonumber
\end{align}
として構成することができる.
この$\bf{10}$表現は$\bm{10}\rightarrow (3^*,1,-\frac{2}{3})\oplus(3,2,\frac{1}{6})\oplus(1,1,1)$
と分解できるので, 次のように当てはめることができる.
\begin{align}
  \overline{\bm{10}}= \frac{1}{\sqrt{2}}\begin{pmatrix}
         0 &  u_3^c & -u_2^c & u_1 & d_1 \\
    -u_3^c &      0 &  u_1^c & u_2 & d_2 \\
     u_2^c & -u_1^c &      0 & u_3 & d_3 \\
    -u_1   &   -u_2 &   -u_3 &   0 & e^c \\
      -d_1 &   -d_2 &   -d_3 &-e^c &   0 \\
    \end{pmatrix}\label{GUT-10rep}
% Note
% 10表現であれば内部に規格化定数を入れる場合があるがここで必要ではないのか確認しなければならない.
% Dec 10: up quark massを出す際に規格化されている必要があるので, この規格化定数は必要.
\end{align}
ただし, $x^c$は$x$を同じカイラリティで荷電共役したものを表している.
この$x$の例として, 左巻きの電子$e^-_L$を考える.
荷電共役変換である$(e^-_L)^c$を考えると, これは右巻きの電子$e_R^-$となるが, 左巻きの陽電子である$e_L^+$と同じである.




%EOF
