はじめに, 指導教官の青木真由美教授には熱心に議論に付き合っていただき感謝いたします.
論文の読み方から書類の書き方, 言葉のお作法, そして物理学に対する姿勢を根本から教えていただきました.
また私の我儘にもかかわらず大統一理論を研究する機会をいただき, 私が手探りで論文を読んでいく中でも辛抱強く議論にお付き合いいただいたことは本当に感謝しています.

また, 武田教授や石渡准教授, 齋川助教にはゼミや普段の議論にお付き合いいただきました.
特に修士課程1年次に行ったすべてのゼミは, 知識を自らのものにすることだけでなく, 知識をどのように伝えて, また議論してゆくのかを身につけることができました.
ゼミで培った能力は論文を読んだり, 物理の議論をする上で何よりも大事な力であると実感しています.
理論物理研究室だからこそ身につけられたような経験をどのような場面であっても活かしていきたいと思います.

理論物理学研究室の仲間の存在にも感謝しています.
村岡先輩, 相澤先輩, 早崎先輩には異なる研究分野であるにもかかわらず, 物理的な質問をされ, 他の分野の立場からどのような疑問が湧き上がるのか, 異なる視点に立つことを学べました.
同期の宮岸君, 澤入君, 巾下君, 佐藤君, 高橋君, そして留学生の牛君, 黄君にはゼミでお世話になりました.
特に同じ学生部屋のメンバーは突発的に議論に付き合ってもらったりして, 非常に助かりました.
後輩の皆さんには, 私が勝手に話しかけに行ったりしても話を聞いてくれたりしてくれて嬉しかったです.
とても良い気分転換になりました\footnote{迷惑でしたらごめんなさい}.

最後に, 私から相談したり心のうちを明かすことなく物理学を学ぶことを半ば勝手に決めたのにも拘らず, 精神的にも経済的にも応援してくれた両親に感謝します.
私が我儘をさせて頂いているのは両親のおかげです.
本当にありがとうございました.
